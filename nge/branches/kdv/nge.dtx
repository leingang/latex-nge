%^^A nge.dtx --- the source file for NGEd package
%^^A $Id$
%^^A
%^^A File nge.dtx is the single source file for the entire NGEd
%^^A package: all the LaTeX classes and styles, all the supporting
%^^A files, and all the documentation is generated from it.  The file
%^^A is structured so that each part in it generates some file or a
%^^A group of files related to each  other in the package, while
%^^A comments in that part provide the documentation.
%^^A
%^^A For general description of the package see section README below, for
%^^A installation instructions see section INSTALL, for license and
%^^A copyright notice see section LICENSE.  Alternatively, processing
%^^A this file through TeX, i.e. running the command 
%^^A       tex nge.dtx
%^^A will extract the content of these sections into files readme.txt,
%^^A install.txt and license.txt, respectively, which can be then
%^^A inspected separately.  This command will also generate other
%^^A files, for details please consult the documentation.
%^^A
%^^A
%^^A
% \iffalse ^^A   SOURCES FOR NGE.INS AND LICENSE.TXT
%^^A The following lines are not part of nge.ins.  They provide
%^^A building blocks for examples, quoted in the documentation,
%^^A explaining the structure of file nge.ins and the part of the
%^^A nge.dtx, controlling extraction of all the pieces. 
%<ext-skel|ext-head>%\iffalse
%<ext-skel>
%^^A
%<*ext-head>
\begingroup
%</ext-head>
%<*ins|ext-docstrip|ext-skel|ext-mock>
\input docstrip.tex
%</ins|ext-docstrip|ext-skel|ext-mock>
%<ext-fullconfig>%<*ins>
%<*ins|ins-config|ext-config|ext-skel|ext-mock>
\keepsilent
%</ins|ins-config|ext-config|ext-skel|ext-mock>
%<ext-fullconfig>%</ins>
%<*ext-config|ext-skel|ext-mock>
\askforoverwritefalse
%</ext-config|ext-skel|ext-mock>
%<ext-fullconfig>%<*ins>
%<*ins|ins-texlive|ext-fullconfig>
\usedir{tex/latex/nge}
%^^A\showdirectory
%</ins|ins-texlive|ext-fullconfig>
%<ext-fullconfig>%</ins>
%<ext-mock-config|ext-skel|ext-mock>\nopreamble\nopostamble
%<ext-preamble>%<*ins>
%<*ins|ins-preamble>
\preamble

%^^A The line above is intentionally left blank for formatting purposes.
%</ins|ins-preamble>
%<ext-preamble>%</ins>
%<ext-preamble>%<*license|ins>
%<*license|ins> ^^A LICENSE and COPYRIGHT NOTICE
Copyright (C) 2003--2009 Matthew Leingang <leingang@courant.nyu.edu>
Copyright (C) 2009 Denis Kosygin <kosygin@courant.nyu.edu>

This work may be distributed and/or modified under the
conditions of the LaTeX Project Public License, either version 1.3
of this license or (at your option) any later version.
The latest version of this license is in:

   http://www.latex-project.org/lppl.txt

and version 1.3 or later is part of all distributions of LaTeX 
version 2005/12/01 or later.

This work has LPPL maintenance status `maintained'.

The Current Maintainer of this work is M. Leingang.

%^^A Add to the line below the names of all generated files or give
%^^A reference to manifest.txt
This work consists of the file nge.dtx and derived files
nge.ins,... (for complete list see file manifest.txt)
%</license|ins>
%<ext-preamble>%</license|ins>
%<ext-preamble>%<*ins>
%<*ins|ins-preamble>
%^^A The line below is intentionally left blank for formatting purposes.

\endpreamble
%</ins|ins-preamble>
%<ext-preamble>%</ins>
%<*ins>
%% Some parts of the source file nge.dtx are concealed from this
%% generation script with the help of <hide> guard.  The correct
%% generation of all the files with the help of docstrip utility
%% relies on the assumption that option `hide' is never issued in
%% generating commands below.
%</ins>
%^^A The next line is added for formatting purposes
%<ext-skel|ext-mock>
%<ext-body>%<*ins>
%<*ins|ins-body|ext-skel|ext-mock>
\generate{
%</ins|ins-body|ext-skel|ext-mock>
%<*ins|ins-body>
%^^A All generation commands for class and style files go here.
  \file{ngemin.cls}{\from{\jobname.dtx}{code,code-cls,ngemin}}
  \file{ngetest.cls}{\from{\jobname.dtx}{code,code-cls,ngetest}}
  \file{ngeexam.cls}{\from{\jobname.dtx}{code,code-cls,ngeexam}}
  \file{ngeprob.sty}{\from{\jobname.dtx}{code,code-sty,ngeprob}}
  \file{ngeproblem.sty}{\from{\jobname.dtx}{code,code-sty,ngeproblem}}
  \file{ngeinit.sty}{\from{\jobname.dtx}{code,code-sty,ngeinit}}
  \file{ngekeys.sty}{\from{\jobname.dtx}{code,code-sty,ngekeys}}
%</ins|ins-body>
%<ext-body>%</ins>
%<ext-body>%<*dist>
%<*dist>
%^^A Generation commands for files in NGEd distribution package,
%^^A including nge.ins, go here.  This block is not included in
%^^A nge.ins
  \file{\jobname.def}{\from{\jobname.dtx}{def}}
  \nopostamble
  \file{\jobname.ins}{\from{\jobname.dtx}{ins,license}}
  \nopreamble\nopostamble
  \file{license.txt}{\from{\jobname.dtx}{license}}
  \file{readme.txt}{\from{\jobname.dtx}{readme}}
  \file{install.txt}{\from{\jobname.dtx}{install}}
%</dist>
%<ext-body>%</dist>
%<ext-body>%<*examples>
%<*examples>
%^^A Files below are not needed for NGEd distribution package.
%^^A They are used as examples in package documentation for
%^^A developers.
  \file{driver-skel.tex}{\from{\jobname.dtx}{driver-skel}}
  \file{driver-body.tex}{\from{\jobname.dtx}{driver-body}}
  \file{driver-2.dtx}{\from{\jobname.dtx}{driver-skel,driver-body}}
  \file{driver-3.dtx}{\from{\jobname.dtx}{driver-skel,driver-body,driver-3}}
  \file{driver-4.dtx}{\from{\jobname.dtx}{driver-skel,driver-body,driver-4}}
  \file{driver-config.tex}{\from{\jobname.dtx}{driver-config}}
  \file{driver-template.dtx}{\from{\jobname.dtx}{driver-skel,driver-config,driver-body,driver-template}}
  \file{driver.dtx}{\from{\jobname.dtx}{driver}}
  \file{\jobname.drv}{\from{\jobname.dtx}{driver,packages,macros}}
  \file{checksum.dtx}{\from{\jobname.dtx}{checksum}}
  \file{stopeventually.dtx}{\from{\jobname.dtx}{stopeventually}}
  \file{nge-1.dtx}{\from{\jobname.dtx}{driver,nge-1,checksum,stopeventually,finale}}
%^^A  \file{ngedoc.dtx}{\from{\jobname.dtx}{ngedoc}}
  \file{hide-example.dtx}{\from{\jobname.dtx}{hide-example}}
  \file{ext-docstrip.dtx}{\from{\jobname.dtx}{ext-docstrip}}
  \file{ext-mock-config.dtx}{\from{\jobname.dtx}{ext-config,ext-mock-config}}
  \file{ext-skel.dtx}{\from{\jobname.dtx}{ext-skel}}
  \file{ext-examples.dtx}{\from{\jobname.dtx}{ext-mock,examples}}
  \file{ins-1.dtx}{\from{\jobname.dtx}{ext-docstrip,ins-config}}
  \file{ins-texlive.dtx}{\from{\jobname.dtx}{ins-texlive}}
  \file{ins-preamble.dtx}{\from{\jobname.dtx}{ins-preamble,license}}
  \file{ins-body.dtx}{\from{\jobname.dtx}{ins-body}}
  \file{ins-tail.dtx}{\from{\jobname.dtx}{ins-tail}}
  \file{ext-head.dtx}{\from{\jobname.dtx}{ext-head}}
  \file{ext-fullconfig.dtx}{\from{\jobname.dtx}{ext-config,ins-config,ext-fullconfig}}
  \file{ext-preamble.dtx}{\from{\jobname.dtx}{ext-preamble,ins-preamble,license}}
  \file{ext-body.dtx}{\from{\jobname.dtx}{ext-body,ins-body,dist,examples}}
  \file{ext-tail.dtx}{\from{\jobname.dtx}{ext-tail,ins-tail}}
%^^A file canary.txt checks for docstrip leaks.
  \file{canary.txt}{\from{\jobname.dtx}{}}
%</examples>
%<ext-body>%</examples>
%<ext-body>%<*ins>
%<*ins|ins-body|ext-skel|ext-mock>
}
%</ins|ins-body|ext-skel|ext-mock>
%<ext-body>%</ins>
%<ext-tail>%<*ins>
%<*ins|ins-tail>
\obeyspaces
\Msg{***************************************************************}
\Msg{*                                                             *}
\Msg{* To finish the installation you have to move all .cls and    *}
\Msg{* .sty files into a directory searched by TeX.                *}
\Msg{*                                                             *}
\Msg{* To produce the documentation run the file nge.dtx           *}
\Msg{* through LaTeX.  For details see file install.txt.           *}
\Msg{*                                                             *}
\Msg{* Happy TeXing                                                *}
\Msg{*                                                             *}
\Msg{***************************************************************}
%^^A The next line is inserted for formatting purposes

%</ins|ins-tail>
%<ext-tail>%</ins>
%<ins|(ins-tail&!ext-tail)>\endbatchfile
%<ext-tail>%<ins>\endbatchfile
%^^A The next line is inserted for formatting purposes.
%<ext-tail>
%<*ext-tail>
\ifx\fmtname\nameofplainTeX
  \expandafter\endbatchfile
\else
  \expandafter\endgroup
\fi
%</ext-tail>
%^^A The following lines are not part of nge.ins.  They provide
%^^A building blocks for examples, quoted in the documentation,
%^^A explaining the structure of file nge.ins and the part of the
%^^A nge.dtx, controlling extraction of all the pieces. 
%<ext-skel>
%<ext-skel|ext-tail>% \fi
% \fi
% \iffalse
%<*hide>
\iffalse %^^A SOURCES for TEXT FILES
%</hide>
%^^A
%^^A
%<*readme> %^^A README
This is file readme.txt for NGEd package.

Installation instructions:
(a copy of these instructions may be found in file instal.txt)
%</readme>
%<install>This is file install.txt with installation instructions for NGEd package.
%<*install|readme> %^^A INSTALL
%^^A the next line is left blank for formatting purposes

To generate the entire package run `tex nge.dtx'
To generate all the class and style files only, run `tex nge.ins'
To generate the documentation run the following commands:
	latex nge.dtx
	makeindex -s gind.ist nge
	makeindex -s gglo.ist -o nge.gls nge.glo
	latex nge.dtx
	latex nge.dtx
Move all .cls and .sty files to a directory, where TeX can find them.
%</install|readme>
%<*hide>
\fi
%</hide>
% \fi
% \iffalse
%^^A
%^^A
% \fi
% \iffalse ^^A     DRIVER
%^^A The following lines form the driver preamble.  They are not part
%^^A of  the driver itself, instead serve as building blocks for
%^^A driver examples, quoted in documentation.
%<driver-4|driver-template>%\iffalse
%<driver-template>%<*driver>
%<driver-3>\iffalse
%^^A Begin driver
%<*driver|driver-skel>
\documentclass{ltxdoc}
%</driver|driver-skel>
%^^A The next line is inserted for formatting purposes
%<driver-skel>
%<*driver|driver-config>
\EnableCrossrefs
\CodelineIndex
\RecordChanges
%</driver|driver-config>
%^^A The next line is inserted for formatting purposes
%<driver|driver-template|nge-1>
%<*driver&packages>
%^^A Draft options
\usepackage{showkeys}
%^^A End of draft options

\usepackage{fancyvrb}

%^^A \usepackage{hyperref}
\usepackage{url}
\newcommand{\texcatalogueentry}[1]%
{\url{http://www.ctan.org/tex-archive/help/Catalogue/entries/#1.html}}
\newcommand{\ctanpackage}[1]%
  {\url{http://mirror.hmc.edu/ctan/macros/latex/contrib/#1}}
\newcommand{\ngedsvnroot}{\url{https://subversive.cims.nyu.edu/mathclinical/}}
\newcommand{\latexpackage}[1]{\textsf{#1}}
\newcommand{\latexclass}{\latexpackage}
\newcommand{\latexfile}[1]{\textsf{#1}}
\newcommand{\docstrip}{\latexpackage{DocStrip}}
\newcommand{\ngefile}[1]{\latexfile{#1}}
\newcommand{\nged}{\latexpackage{NGEd}}

\usepackage[dvistyle]{todonotes}
\newcommand{\fixme}[1][]{\todo{FIX ME! #1}}

\usepackage{svninfo}
%</driver&packages>
%<*driver&macros>
%</driver&macros>
%<*driver|driver-skel>
\begin{document}
%</driver|driver-skel>
%^^A The next line is inserted for formatting purposes
%<driver-skel&!driver-body>
%<*driver|driver-body>
  \DocInput{\jobname.dtx}
%</driver|driver-body>
%<*driver|driver-skel>
\end{document}
%</driver|driver-skel>
%^^A End driver
%^^A
%^^A The following lines form the driver postamble.  They are not part
%^^A of  the driver itself, instead serve as building blocks for
%^^A driver examples, quoted in documentation.  In particular, they
%^^A provide matching \fi's for each \iffalse, placed in drivers'
%^^A preamble. 
%<driver-3>\fi
%<driver-template>%</driver>
%<driver-4|driver-template>%\fi
% \fi
%\iffalse  ^^A CHECKSUM
%<*checksum>
%^^A The next line is inserted for formatting purposes
%<nge-1>%
%\fi
%% \CheckSum{0}
%% \CharacterTable
%%  {Upper-case    \A\B\C\D\E\F\G\H\I\J\K\L\M\N\O\P\Q\R\S\T\U\V\W\X\Y\Z
%%   Lower-case    \a\b\c\d\e\f\g\h\i\j\k\l\m\n\o\p\q\r\s\t\u\v\w\x\y\z
%%   Digits        \0\1\2\3\4\5\6\7\8\9
%%   Exclamation   \!     Double quote  \"     Hash (number) \#
%%   Dollar        \$     Percent       \%     Ampersand     \&
%%   Acute accent  \'     Left paren    \(     Right paren   \)
%%   Asterisk      \*     Plus          \+     Comma         \,
%%   Minus         \-     Point         \.     Solidus       \/
%%   Colon         \:     Semicolon     \;     Less than     \<
%%   Equals        \=     Greater than  \>     Question mark \?
%%   Commercial at \@     Left bracket  \[     Backslash     \\
%%   Right bracket \]     Circumflex    \^     Underscore    \_
%%   Grave accent  \`     Left brace    \{     Vertical bar  \|
%%   Right brace   \}     Tilde         \~}
%\iffalse
%</checksum>
%\fi
%^^A
%^^A The actual code and documentation go below this line.
% \svnInfo $Id$ %^^A
% \providecommand*{\filename}{}
% \renewcommand{\filename}{\jobname.dtx}
% \providecommand*{\filedate}{}
% \renewcommand*{\filedate}{\svnYear/\svnMonth/\svnDate}
% \providecommand*{\fileversion}{}
% \renewcommand*{\fileversion}{development version, revision \svnInfoRevision}
%^^A \ProvideFile{\filename}[\filedate\space\fileversion]
% \title{\nged: A Next Generation Educational Bundle\\ \fileversion}
% \author{Matthew Leingang\\leingang@courant.nyu.edu
%   \and Denis Kosygin\\kosygin@courant.nyu.edu}
% \renewcommand{\date}{\svnToday}
% \maketitle
% \begin{abstract}
%   The \nged{} bundle is a set of \LaTeXe classes and packages,
%   providing a unified typographic support for the entire line of
%   educational materials: from tests and solutions to course
%   assessments.  Its naturally oriented toward mathematics courses
%   but it is not specific to them.
% \end{abstract}
%
% \tableofcontents
%
% \section{Introduction}
%
%
% \subsection{Goals}
%\todo{Rewrite!}
% The \nged{}bundle classes and packages should be
%
% \begin{itemize}
%   \item \emph{beautiful}
%   \item \emph{useful} 
%   \item \emph{flexible} items should be configurable
%   \item \emph{modular} This is why the several different packages for different uses
%   \item \emph{lazy} We will use other packages when they are useful
% \end{itemize}
%
% \subsection{Other classes which do this kind of thing}
%\todo{Expand!}
% Look at examdesign, mathexam, and exam.
% 
% \subsection*{Introduction}
%
% \subsubsection*{Conventions, adopted in is this manual}
% The word ``package'' is ambiguous and its meaning varies with the
% context.  In this manual it may mean any of the following:
% \begin{description}
% \item[\LaTeX{} package:] a collection of \LaTeX{} styles and classes,
%   designed to work well together and which may depend on each other.
%   Files, comprising a \LaTeX{} package are usually located in a
%   common directory of a \LaTeX{} distribution tree.  In particular,
%   all files of \LaTeX{} package \nged, described in this
%   manual share are located in directory \ngefile{nge}.
% \item[Distribution package:] a \LaTeX{} package together with its
%   documentation and other supporting files (e.g. a list of files
%   with their md5 checksums).  Distribution packages are frequently
%   taylored to one of popular \TeX{} distributions, such as teTeX or
%   liveTeX and most of them are available for download in archived
%   forms at large Internet repositories, such as CTAN.\fixme{CTAN address here}
% ^^A When this package is released and accepted to CTAN, uncommment
% ^^A the following lines and insert the appropriate address.
% \iffalse (this is meta-comment)
%  In particular, the current version of \nged, documented in
%  this manual, is available for download on CTAN
%  and various CTAN mirrors (see \ctanpackage{}).
%   This distribution package does not actually contain any \LaTeX{} class or
%   style files, they are generated automatically during package installation.
% \fi
% \item[Source package:] a collection of files and tools necessary for
%   the production and further development of the corresponding
%   \LaTeX{} package together with its supporting materials.  Source
%   package may also include various configuration files for tools,
%   package maintainers and developers use during the development process.
% \end{description}
% Most of the time this ambiguity does not lead to confusion, since
% the intented meaning of word ``package'' is clear from the context.
% In section \ref{sec:user} ``package'' usually means the \LaTeX{}
% \nged{} package and in section \ref{sec:admin} ``package''
% usually means the distribution package \ngefile{nge}.
%
% \subsection{History and changes}
%
% M.\ Leingang first wrote the exam class in 2003 and has developed it
% in stages since then.  Other pieces have been developed along the
% way, too.  In 2009 D.\ Kosygin refactored the package and its
% documentation.  Version 3.0 has been released to public and uploaded
% on CTAN on\fixme{date here}
%^^A insert date here!
%
%
% \section{Using \nged}
% \label{sec:user}
% 
% \subsection{Regenerating user manual}
% \label{sec:user-manual}
%
% 
% \subsection{Class \ngefile{ngemin.cls}}
% \label{sec:user-ngemin}
%
% This class defines the common theme for all the documents produced
% with the help of \nged.  All other classes in this package
% load it and add their own changes to document appearence.
%
% \section{Administration of \nged}
% \label{sec:admin}
%
% This section discusses \nged{} distribution package as
% described in the introduction of this manual.  The structure of this
% package follows popular conventions, adopted in most of \LaTeX{}
% packages of \TeX{}\ Live distribution.  \LaTeX{} code for the entire
% package together with examples and supporting documentation is
% contained in a single file \ngefile{nge.dtx}.
%
% \subsection{Installation}
% \label{sec:admin-installation}
%
% \subsection{Regenerating manuals}
% \label{sec:admin-installation-manuals}
%
% \subsection{Configuration}
% \label{sec:admin-configuration}
%
%\iffalse
%<*stopeventually>
%^^A The next line is added for formatting purposes
%<nge-1>%
%\fi
%% \StopEventually{\PrintChanges\PrintIndex}
%\iffalse
%</stopeventually>
%\fi
%
%
% \section{\LaTeX{} code}
% \label{sec:code}
%
%\iffalse
%<*code>
%\fi
% All the classes and styles in this package rely on \LaTeXe.  No
% support for earlier versions of \LaTeX{} is provided.
%    \begin{macrocode}
\NeedsTeXFormat{LaTeX2e}
%<ngemin>\ProvidesClass{ngemin}
%<ngeinit>\ProvidesPackage{ngeinit}
%<ngeunit>\ProvidesPackage{ngeunit}
%<ngetest>\ProvidesClass{ngetest}
%<ngeexam>\ProvidesClass{ngeexam}
%<ngequiz>\ProvidesClass{ngequiz}
%<ngeprob>\ProvidesPackage{ngeprob}
%<ngeproblem>\ProvidesPackage{ngeproblem}
%<ngekeys>\ProvidesPackage{ngekeys}
%<def>\ProvidesFile{nge.def}
%^^A  [\filedate \fileversion \filedescription]
%    \end{macrocode}
% The following package contains definitions, necessary for common
% options' declaration in \nged.
%    \begin{macrocode}
%<!ngeinit>\RequirePackage{ngeinit}
%    \end{macrocode}
% We reserve prefix |NGE@| for internal commands in \nged{} classes
% and styles.
% 
%\iffalse
%</code>
%\fi
%
% \subsection{File \ngefile{nge.def}}
% \label{sec:code-def}
%
% \iffalse
%<*def>
% \fi
% This file contains information, common for all the files in \nged{}
% package.  First we introduce necessary commands, in case they are
% not defined yet.
%    \begin{macrocode}
\providecommand*{\fileversion}{--}
\providecommand*{\filedate}{--}
\providecommand*{\filedescription}{--}
%    \end{macrocode}
% Now let us supply the values.
%    \begin{macrocode}
\renewcommand*{\fileversion}{--}
\renewcommand*{\filedate}{--}
%    \end{macrocode}
% File description requires more care.
%
% \iffalse
%</def>
% \fi
% \subsection{File \ngefile{ngeinit.sty}}
% \label{sec:code-ngemin}
%
% \iffalse
%<*ngeinit>
% \fi
% This is the internal package for \nged, which makes sure, that
% everything needed for correct declaration of options is defined and
% defined only once.
% 
% Switches for boolean options.
%    \begin{macrocode}
\RequirePackage{ifthen}
\newboolean{points}
%    \end{macrocode}
% 
%\iffalse
%</ngeinit>
%\fi
% \subsection{File \ngefile{ngemin.cls}}
% \label{sec:code-ngemin}
%
%\iffalse
%<*ngemin>
%\fi
% This class defines the common theme for all the documents produced
% with the help of \nged.  All other classes in this package
% load it and add their own changes to document appearence.
%    \begin{macrocode}
\RequirePackage{ifthen}
%    \end{macrocode}
% It is derived from \LaTeX{} class \latexfile{article.cls}
% to which we pass all the unknown options
%    \begin{macrocode}
\DeclareOption*{\PassOptionsToClass{\CurrentOption}{article}}
%    \end{macrocode}
% Now it is time to load the class and process all options given to it
%    \begin{macrocode}
\ExecuteOptions{}
\ProcessOptions
%    \end{macrocode}
%    \begin{macrocode}
\LoadClass{article}
\RequirePackage{fancyhdr}
%^^A\RequirePackage{titling}
%    \end{macrocode}
% For the moment we'll do quick mockup of some features, without
% paying attention to cleanness of implementation and detail.  We'll
% polish the code later.
%
% Page style with namehead
%    \begin{macrocode}
\newcommand{\ps@NGEnamehead}{%
  \renewcommand{\@oddhead}{%
    \begin{minipage}{\textwidth}
      \normalfont
      Name: \hrulefill\
      Section:  \rule{0.5in}{0.4pt}
      \\ \\
      \noindent
      Student Id: \rule{1.5in}{0.4pt}\
      Email: \hrulefille
    \end{minipage}}
  \renewcommand{\@evenhead}{\@oddhead}
  \renewcommand{\@oddfoot}{%
    Signature: \hrulefill}
  \renewcommand{\@evenfoot}{}}
%    \end{macrocode}

% First let's introduce some basic course information and then we'll
% deal with appearances.
% |\NGEcourse| sets the course name and short course name to be used
% in the header.
%    \begin{macrocode}
\newcommand{\NGE@shortcoursename}{}
\newcommand{\NGE@coursename}{}
\newcommand{\NGE@course}{}
\newcommand{\NGEcourse}{\@dblarg\NGE@@course}
\def\NGE@@course[#1]#2{
  \renewcommand{\NGE@shortcoursename}{#1}
  \renewcommand{\NGE@coursename}{#2}
  \renewcommand{\NGE@course}{%
    \Large%
    \begin{tabular}[t]{c}%
      \NGE@coursename%
    \end{tabular}%
    \par\vskip 0.75em}}
%    \end{macrocode}
% \begin{macro}{NGE@author}
% Provide for possibility when author is not given
%    \begin{macrocode}
\newcommand{\NGE@author}{}
\newcommand{\noauthor}{\renewcommand{\NGE@author}{}}
\let \NGE@oldauthor \author
\renewcommand{\author}[1]{
 \NGE@oldauthor{#1}
 \renewcommand{\NGE@author}{\vskip 1.5em%
   {\large%
     \lineskip .75em%
     \begin{tabular}[t]{c}%
       \@author%
     \end{tabular}\par}}}
%    \end{macrocode}
% \end{macro}
% Let us deal with the titling commands.
% \begin{macro}{maketitle}
% Here we reimplement |\maketitle| command from
% \latexclass{article.cls} by making it accept and typeset course
% information.  Could have used package \latexpackage{titling}, but
% right now it has a bug I'd rather wait until it is fixed.  However
% we shall use try to use notations, consistent with it.
%    
% We have to redefine |\maketitle| command in order to incorporate
% course information into it.  The definition depends on whether a
% separate title page is produced (this is the default for exams).  We
% use the switch |@titlepage| defined in \latexclass{article} class
% and reproduce most of the code from \latexfile{article.cls}
% verbatim.
%    \begin{macrocode}
\newlength{\NGE@pretitleskip}
\newlength{\NGElowertitle}
\setlength{\NGElowertitle}{0em}
%    \end{macrocode}
%    \begin{macrocode}
\ifthenelse{\boolean{@titlepage}}
%    \end{macrocode}
% we are in the title page.
%    \begin{macrocode}
{\renewcommand{\maketitle}{\begin{titlepage}%
  \let\footnotesize\small
  \let\footnoterule\relax
  \let \footnote \thanks
  \setlength{\NGE@pretitleskip}{60\p@}
  \addtolength{\NGE@pretitleskip}{\NGElowertitle}
  \null\vfil
  \vskip \NGE@pretitleskip%
  \begin{center}%
    {\LARGE \@title \par}%
    \vskip 3em%
    {\large
     \lineskip .75em%
      \begin{tabular}[t]{c}%
        \@author
      \end{tabular}\par}%
      \vskip 1.5em%
    {\large \@date \par}%       % Set date in \large size.
  \end{center}\par
  \@thanks
  \vfil\null
  \end{titlepage}%
  \setcounter{footnote}{0}%
  \global\let\thanks\relax
  \global\let\maketitle\relax
  \global\let\@thanks\@empty
  \global\let\@author\@empty
  \global\let\@date\@empty
  \global\let\@title\@empty
  \global\let\title\relax
  \global\let\author\relax
  \global\let\date\relax
  \global\let\and\relax}}
{\renewcommand{\maketitle}{\par
  \begingroup
    \renewcommand\thefootnote{\@fnsymbol\c@footnote}%
    \def\@makefnmark{\rlap{\@textsuperscript{\normalfont\@thefnmark}}}%
    \long\def\@makefntext##1{\parindent 1em\noindent
            \hb@xt@1.8em{%
                \hss\@textsuperscript{\normalfont\@thefnmark}}##1}%
    \if@twocolumn
      \ifnum \col@number=\@ne
        \@maketitle
      \else
        \twocolumn[\@maketitle]%
      \fi
    \else
      \newpage
      \global\@topnum\z@   % Prevents figures from going at top of page.
      \@maketitle
    \fi
    \thispagestyle{plain}\@thanks
  \endgroup
  \setcounter{footnote}{0}%
  \global\let\thanks\relax
  \global\let\maketitle\relax
  \global\let\@maketitle\relax
  \global\let\@thanks\@empty
  \global\let\@author\@empty
  \global\let\@date\@empty
  \global\let\@title\@empty
  \global\let\title\relax
  \global\let\author\relax
  \global\let\date\relax
  \global\let\and\relax}}
%    \end{macrocode}
% here we are not in the title page.  Since the actuall typesetting is
% done by |\@maketitle|, we just redefine it.
%    \begin{macrocode}
\def\@maketitle{%
  \setlength{\NGE@pretitleskip}{2em}
  \addtolength{\NGE@pretitleskip}{\NGElowertitle}
  \newpage
  \null
  \vskip \NGE@pretitleskip%
  \begin{center}%
    \let \footnote \thanks
    \NGE@course%
    {\LARGE \@title \par}%
    \NGE@author%
    \vskip 1em%
    {\large \@date}%
  \end{center}%
  \par
  \vskip 1.5em}
%    \end{macrocode}
% \end{macro}
%
%\iffalse
%</ngemin>
%\fi
%
% \subsection{Package \latexclass{ngeunit}}
% \label{sec:code-ngeunit}
%
%\iffalse
%<*ngeunit>
%\fi
%
% This package supplies the logical structure and the common theme for
% all other classes.
%
% The entire \nged{} package is centered around the concept of a
% course as the single unit.  This package defines various commands
% for specifying and accessing various course properties.
%
%
%
%
%
%
%\iffalse
%</ngeunit>
%\fi
%
% \subsection{Class \latexclass{ngetest} and its derivatives}
% \label{sec:code-ngetest}
%
%
%\iffalse
%<*ngetest>
%\fi
% 
% This is the internal package, where all the necessary definitions
% are given.  The derived classes, such as \ngefile{ngeexam} and
% \ngefile{ngequiz}, just set appropriate defaults before loading this
% class.
%
% We use package \latexpackage{ifoption} from \latexpackage{amsrefs}
% collection.  This package provides useful command |\IfOption| and
% also makes handling of mutually exclusive options more consistent.
%    \begin{macrocode}
\RequirePackage{ifoption}
%    \end{macrocode}
% 
%
%    \begin{macrocode}
\DeclareExclusiveOptions{pages,bluebook}
%    \end{macrocode}
% Options below are not really boolean options.  It would be more
% appropriately to call them \emph{extensions}.  Their purpose is to
% extend the functionality of the package by adding and modifying
% various macros, but once they are processed it is not easy to undo
% their effect.  But since it is possible to test whether such an
% option was given to the class, it is possible to emulate boolean
% switches with their help too.
%    \begin{macrocode}
\DeclareBooleanOption{solutions}
\DeclareBooleanOption{rubrics}
%    \end{macrocode}
% We also define a genuine boolean switch, which makes it possible to
% turn processing points on and off explicitely.  Emulating it with
% extension switch would be awkward, since the default is to assign
% points to problems.
%    \begin{macrocode}
\DeclareOption{points}{\setboolean{points}{true}%
  \PassOptionToPackage{\CurrentOption}{ngeproblem}}
\DeclareOption{nopoints}{\setboolean{points}{false}%
  \PassOptionToPackage{\CurrentOption}{ngeproblem}}
%    \end{macrocode}
% All unknown options are passed to class \latexpackage{article}.
%    \begin{macrocode}
\DeclareOption*{\PassOptionToClass{\CurrentOption}{article}}
%    \end{macrocode}
% By default points are assigned to each problem
%    \begin{macrocode}
\ExecuteOptions{points}
\ProcessOptions\relax
\LoadClass{article}
%    \end{macrocode}
%    \begin{macrocode}
\ProcessExclusiveOptions\relax
%    \end{macrocode}
%    \begin{macrocode}
\IfOPtion{bluebook}{}{}
%    \end{macrocode}
%    \begin{macrocode}
\IfOPtion{pages}{}{}
%    \end{macrocode}
%    \begin{macrocode}
\IfOPtion{solutions}{}{}
%    \end{macrocode}
% Here we need to consider too possibilities: whether solutions to the
% test are provided or not.
%    \begin{macrocode}
\IfOPtion{rubrics}{%
  \IfOption{solutions}{}{}}{}
%    \end{macrocode}
% Finally we load package \ngefile{ngeproblem} which inherits the
% options from \ngefile{ngetest}.
%    \begin{macrocode}
\RequirePackage{ngeproblem}
%    \end{macrocode}
% \subsubsection{Table of problems}
% \label{sec:code-ngetest-top}
%
% \begin{macro}{\tableofproblems}
%  This is done like a table of contents with |\@starttoc| command.
%    \begin{macrocode}
\newcounter{totalpoints}
\newcommand{\tableofproblems}{\@starttoc{top}}
%    \end{macrocode}
% Internal environment |\NGE@tableofproblems| generates the top and
% the bottom of the table of problems.
%    \begin{macrocode}
\newenvironment{NGE@tableofproblems}{%
  \setcounter{totalpoints}{0}
  \begin{center}
    \begin{tabular}{|c|c|c|}\hline
      Problem& Possible & Points\\
      Number & Points & Earned\\\hline
    }{%
      \hline
      \NGE@problemstrut
      Total &\arabic{totalpoints}&\\\hline
      \end{tabular}
\end{center}}
%    \end{macrocode}
% This internal macro generates a line of the table of problems
%    \begin{macrocode}
\newcommand{\NGE@problem}[2]{%
  \addtocounter{totalpoints}{#2}
  \NGE@problemstrut
  #1 & #2 & \\\hline
}      
%    \end{macrocode}
% To make the table large enought to write in with red pen, this
% internal macro genrates some vertical space
%    \begin{macrocode}
\pgfqkeys{/NGE/table of problems}{row height/.initial=5ex,
row depth/.initial=2ex}
\newcommand{\NGE@problemstrut}{%
\rule[-\pgfkeysvalueof{/NGE/tableofproblems/row depth}]{0pt}%
{\pgfkeysvalueof{/NGE/tableofproblems/row height}}}
%    \end{macrocode}
% \todo{It might be nicer to specify the \emph{total} height of the
% row, with the problem number centered in the row.}
%
% Now we add hooks to |\begin{document}| and |\end{document}| to open
% the table of problems file, write the top, write the bottom and
% close it.
%    \begin{macrocode}
\AtBeginDocument{%
  \addtocontents{top}{\protect\begin{NGE@tableofproblems}}}
%    \end{macrocode}
% For the end hook we'd like to do
% \begin{Verbatim}
%   \addtocontents{top}{\protect\end{NGE@tableofproblems}}
% \end{Verbatim}
% but for some reason that does not do anything.  So we go one step
% deeper in the abstraction and copy the definition of |\addtocontents|
%    \begin{macrocode}
\AtEndDocument{%
  \immediate\write\@auxout{\string\@writefile{top}%
  {\string\end{NGE@tableofproblems}}}}
%    \end{macrocode}
% 
% \end{macro}
% 
%
%
%
%
%
%
%
%\iffalse
%</ngetest>
%\fi
%
% Classes \latexclass{ngeexam} and \latexclass{ngequiz} do little
% other than selecting appropriate defaults.
%
% \subsection{File \ngefile{ngeexam.cls}}
% \label{sec:code-ngeexam}
%
%\iffalse
%<*ngeexam>
%\fi
% 
% Class \ngefile{ngeexam} does little other than selecting
% appropriated defaults and loading other packages.
%    \begin{macrocode}
\DeclareOption*{\PassOptionToClass{\CurrentOption}{ngetest}}
%    \end{macrocode}
% 
%
%    \begin{macrocode}
\ExecuteOptions{pages}
\ProcessOptions\relax
\LoadClass{ngetest}
%    \end{macrocode}
%\iffalse
%</ngeexam>
%\fi
%
% \subsection{File \ngefile{ngeprob.sty}}
% \label{sec:code-ngeprob}
%
%\iffalse
%<*ngeprob>
%\fi
%
%    \begin{macrocode}
\ExecuteOptions{}
\ProcessOptions
%    \end{macrocode}
%
%
%
%
%
%
%
%\iffalse
%</ngeprob>
%\fi
%
%
% \subsection{File \ngefile{ngeproblem.sty}}
% \label{sec:code-ngeproblem}
%
%\iffalse
%<*ngeproblem>
%\fi
% This package provides tools for |problem| environment.  It is
% designed to that it could be used separately from the rest of
% \nged{} package.
%
%    \begin{macrocode}
\RequirePackage{ifoption}
%%\RequirePackage{pgfopts}
%    \end{macrocode}
%    \begin{macrocode}
\DeclareOption{points}{\setboolean{points}{true}}
\DeclareOption{nopoints}{\setboolean{points}{false}}
%    \end{macrocode}
%
%    \begin{macrocode}
\DeclareBooleanOption{solutions}
\DeclareBooleanOption{rubrics}
%    \end{macrocode}

%    \begin{macrocode}
\ExecuteOptions{points}
\ProcessOptions\relax
\ProcessExclusiveOptions\relax
%    \end{macrocode}
%
% \subsubsection{Problems, solutions and hints}
% \label{sec:code-ngeproblem-problems}
%    \begin{macrocode}
\RequirePackage{pgfkeys}
\RequirePackage{ifthen}
%    \end{macrocode}
% The documentation for \latexpackage{amsthm} says that it should be
% loaded \emph{after} \latexpackage{amsmath} is loaded.  So we
%    \begin{macrocode}
\RequirePackage{amsmath}
\RequirePackage{amsthm}
%    \end{macrocode}
% Initially we define just one environment |problem| and also provide
% for user options to adjust its behaviour as desired.
% \begin{macro}{problem}
% Problem number is kept track of in counter |problem|, counter
% |problempoints| contains the number of points assigned to the
% problem.  Default value is 0.
%    \begin{macrocode}
\newcounter{problem}
\newcounter{problempoints}
%    \end{macrocode}
% The problem itself is setup with the help of \latexpackage{asmthm}
% package.  \fixme{provide user command to change it!}.
%    \begin{macrocode}
\pgfkeys{/NGE/problem/.cd,
body font/.initial=\normalfont,
header font/.initial=\bfseries}
\newcommand{\NGE@problembodyfont}{\normalfont}
\newcommand{\NGE@problemheaderfont}{\bfseries}
\newtheoremstyle{NGE@problem}
{}% Space above
{}% Space below
{\NGE@problembodyfont}% problem body font
{}% indent amount (empty=no indent, \parindent = normal paragraph indent)
{\NGE@problemheaderfont}% problem header font
{.}% Punctuation after theorem head
{ }% Space after theorem head
% ({ }=normal interword space, \newline=line break).
{}% Head specification (empty means normal
\newcommand{\NGE@problemname}{Problem}
\theoremstyle{NGE@problem}
\newtheorem{NGE@problem}[problem]{\NGE@problemname}
\theoremstyle{plain}
%    \end{macrocode}
% Now to the the working horse: environment |problem|.  Initially it
% does very little.  It takes one optional argument --- a number of
% points for the problem, and, depending on the setting of the switch
% |points| it passes it to |NGE@problem| or merely swallows it.  
%    \begin{macrocode}
\newenvironment{problem}[1][]
%    \end{macrocode}
% First we process all given options in order they were given.
%    \begin{macrocode}
{\setcounter{problempoints}{0}
  \pgfqkeys{/NGE/problem/options}{#1}
  \ifthenelse{\boolean{points}}{\begin{NGE@problem}[\theproblempoints]}
    {\begin{NGE@problem}}}
{\end{NGE@problem}}
%    \end{macrocode}
% Redefine macro |\theproblempoints| to expand number n to ``n points''.
%    \begin{macrocode}
\renewcommand{\theproblempoints}{%
\arabic{problempoints} point\ifthenelse{\value{problempoints} = 1}{}{s}}
%    \end{macrocode}
%    \begin{macrocode}
\pgfkeys{/NGE/problem/options/points/.cd,
  .value required, .code=\setcounter{problempoints}{#1}}
%    \end{macrocode}
% \end{macro}
% Option |nopoints| disables points showing.  It accepts one optional
% argument to allow to comment out assignment |points=10| as
% |nopoints=10|, which then is ignored.
%    \begin{macrocode}
\pgfkeys{/NGE/problem/options/nopoints/.code=\setboolean{points}{false}}
%    \end{macrocode}
% Notice that option assignment |20,nopoints| still locally sets the
% counter |problempoints| to 20, but no points values is typeset in
% problem header.  Right now I have no idea if such behaviour is
% useful for anything, but who knows, may be some hack later on will
% justify it.
%
% Finally, in order to give ability to turn on point typesetting on
% per problem basis, we provide boolean option |problemwithpoints|,
% which locally sets the boolean switch |points| in each problem.
% Now let's deal with unknown options.  For the moment we assume that
% any unknown option is a nonegative number, and interpret it as the
% desired number of points for this problem.
%    \begin{macrocode}
\pgfkeys{/NGE/problem/options/.unknown/.code
    = \setcounter{problempoints}{\pgfkeyscurrentname}}
%    \end{macrocode}
% 
% \begin{macro}{remark}
%    \begin{macrocode}
\theoremstyle{remark}
\newtheorem*{remark}{Remark}
\newtheorem*{note}{Note}
%    \end{macrocode}
% \end{macro}
% \begin{macro}{solution}
%    \begin{macrocode}
\newenvironment{solution}[1][Solution]{\begin{proof}[#1]}{\end{proof}}
% \end{macro}
% \begin{macro}{answer}
%    \begin{macrocode}
\theoremstyle{remark}
\newtheorem*{answer}{Answer}
% \end{macro}
% 
% Switch back to default theorem style
%    \begin{macrocode}
\theoremstyle{plain}
%    \end{macrocode}
% \end{macro}
%    \begin{macrocode}
\IfOption{solutions}{}{}
%    \end{macrocode}
%    \begin{macrocode}
\IfOption{rubrics}{}{}
%    \end{macrocode}
%\iffalse
%</ngeproblem>
%\fi
%
% \subsection{Package \latexclass{ngekeys}}
% \label{sec:code-ngekeys}
%
%\iffalse
%<*ngekeys>
%\fi
%
% This package supplies the interface for keys processing with the   
% help of \latexpackage{pgfkeys}.
%    \begin{macrocode}
\RequirePackage{pgfkeys}
%    \end{macrocode}
% \nged{} package reserves the tree of names with root |/NGE|.
% Subtree |/NGE/options| provides support for various package options,
% subtree |/NGE/errors| holds error messages.
%
% \subsubsection{Options support}
% \label{sec:code-ngekeys-options}
%
% \begin{macro}{\DeclareNGEBooleanOption}
\newboolean{testboolean}
\pgfkeys{/test/boolean key/.is if=testboolean}
\pgfkeys{/test/boolean key=true}
\newcommand{\DeclareNGEBooleanOption}[2]{%
  \newboolean{NGE@#1}
  \pgfkeys{/NGE/#2/#1/key/.is if=NGE@#1,
    /NGE/#2/#1/.cd,
    /NGE/#2/#1/.is choice,
    /NGE/#2/#1/.default=true,
    true/.code={\setboolean{#1}{true}},
    yes/.code={\setboolean{#1}{true}},
    false/.code={\setboolean{#1}{false}},
    no/.code={\setboolean{#1}{false}},
    /NGE/#2/no#1/.code={\setboolean{#1}{false}}
}}
% \end{macro}
%
% \subsubsection{Error messages}
% \label{sec:code-ngekeys-errors}
%
% \begin{macro}{NGEerror}
% Command |\NGEDefineError|
% defines an error  and stores error message 
%    \begin{macrocode}
\newcommand{\NGEDefineError}[3]{%
  \pgfqkeys{/NGE/errors/#1}{.initial=#2,help message=#3}
}
%    \end{macrocode}
% We define common handler for all errors
%    \begin{macrocode}
\pgfkeys{/handlers/.NGEerror/.code
  =\PackageError{NGEd}{%
    \pgfkeysvalueof{/NGE/errors/\pgfkeyscurrentname}}{%
    \pgfkeysvalueof{/NGE/errors/\pgfkeyscurrentname/help message}}}}
\newcommand{\NGEerror}[1]{\pgfkeys{/NGE/errors/#1}}
%    \end{macrocode}
%    
% \end{macro}
%\iffalse
%</ngekeys>
%\fi
%
% \section{Source}
% \label{sec:source}
% 
% This section discusses the collection of files and tools used in
% generation and development of \nged{} bundle, that is, in
% terms of conventions in the beginning of this manual, the
% \emph{source package} \nged.  We shall refer to files in
% \nged{} source package as \nged{} \emph{sources} or
% simply \emph{the sources}.
%
% \subsection{Main source file: \ngefile{nge.dtx}}
% \label{sec:source-dtx}
%
% 
% File \ngefile{nge.dtx} is the single source file for the entire
% \nged{} package: all the \LaTeX{} classes and styles, all the
% supporting files, and all of the documentation is generated from it.
% The consists mostly of \LaTeX{} code together with copious comments
% which provide the source for the documentation.  The file is
% structured so that each part in it generates some file or a group of
% files related to each other in the package.
%
%
%
% \subsection{Generation of \LaTeX{} files and other files:
%   \ngefile{nge.ins}}
% \label{sec:source-ins}
%
% In order to avoid duplication of effort, most of the package files
% are generated from \ngefile{nge.dtx} by running various programs on
% it. In particular, running \TeX{} on \ngefile{nge.dtx} generates
% \ngefile{nge.sty} and all the class and style files.  Similarly,
% running \TeX{} on \ngefile{nge.ins} generates all the class and
% style files, but not \ngefile{nge.ins} itself, in order to avoid
% vicious loops.  Setting the appropriate file structure takes some
% effort, but once it is in place, adding to it is not very difficult.
%
%
% \subsection{Documentation}
% \label{sec:source-documentation}
%
% \LaTeX{} code for each class and style file must be placed in a
% separate subsection of section \ref{sec:code} together with detailed
% explanations of its structure.  The title of this section is just
% the name of this class or style file.  In addition there must be a
% subsection of section \ref{sec:user} with the same title, discussing
% with examples how to use it.
%
% \section{Development notes and history}
% \label{sec:dev}
%
% Understanding the structure of a mature
% software package frequently is not an easy task.
% This section documents the entire design of \nged{} package and the
% history of its development.   After
% several cycles of development such a package may consist of several,
% (sometimes many) sophisticated parts interacting is subtle and
% complex.  I\footnote[1]{\label{footnote:1}DK} find it helpful to
% document the entire process of development from the scratch, stage
% by stage, introducing components of the design and implementation
% one by one.  This way it is easier to comprehend (or recall after
% prolonged interruption) the underlying structure of the package and
% the logic of its evolution.  This section is written for my own
% benefit in the first place and bearing in mind a possibility that
% someone else may find it useful too.  The exposition here is akin to
% an exposition of a physical theory.  It does not reflect the
% chronological order of development of ideas and tools, but instead
% offers a coherent path, leading from one idea to another.
%
% \subsection{The purpose of \nged}
% \label{sec:dev-overview}
%
% I\footnotemark[1]\fixme view \nged{} as a tool of teacher's personal
% toolkit in an environment where teaching happens on an industrial
% scale: a large school, a college with large classes, etc\dots.  This
% tool provides typographic support for the entire lifetime of a
% course taught, from its development stage through periodic
% assesments to a complete bank of materials: assignments, tests,
% handouts for students, notes for teachers and internal
% documentation.  In an ideal teaching environment \nged{} will be
% complemented by a database with a bank of questions, problems,
% tasks, and other materials, crossreferenced by subjects, topics,
% keywords, difficulty level and so on, and teaching support system,
% which automates as much as possible daily routine tasks such as
% distribution and administration of tests, grade processing and
% anything else which which may be reliably delegated to machines and
% computers.
%
% \nged{} should make it possible to concentrate on development of
% course structure, preparation of its materials without worrying too
% much about uniformity and consistency of the appearence of all the
% documents.  For example, preparation of exams, their solutions for
% students, rubric notes for graders, postmortem discussions for
% instructors and assessments for course designers could be possible
% with selection of a few virtual switches applied to the contents of
% the course, which integrates the printed output with other course
% materials under the same typographic theme.
%
% On the other hand \nged{} must be flexible enough to allow
% customization of such themes sa desired.  In addition the package
% should provide a set of sensible defaults which work well out of the
% box and supply an easily extensible theme, which may serve as a
% basis for derivation of other themes, taylored to particular needs
% of course designer.
%
% \subsection{Design and implementation decisions}
% \label{sec:dev-design}
%
% \nged{} grew naturally out of design of tests and course materials
% in mathematical courses, and there is a widely followed tradition of
% using \LaTeX{} and other \TeX{}-derived programs for typesetting
% mathematics.  It is then a straightforward decision to develop
% \nged{} as a \LaTeX{} package at the same time keeping an eye for a
% possibility of extending it later to other \TeX-breeds, such as
% ConTeXt.  
%
% Package \nged{} will be developed following standards and
% conventions of \TeX{}\ Live
% distribution\footnote{\url{http://www.tug.org/texlive/}} with the
% ultimate goal of making it available for public download on the
% Comprehensive TeX Archive Network\footnote{CTAN,
%   \url{http://www.ctan.org}}.  \nged{} should integrate well with
% \TeX{}\ Live distribution.  In particular \nged{} in its default
% configuration must not rely on any package outside for this
% distribution neither in application of its classes and style files,
% nor in generation of package documentation.
% 
% The released version of \emph{\nged{} distribution package},
% intended for upload on CTAN, should consist of two main files:
% \ngefile{nge.dtx} and \ngefile{nge.ins}, supplemented with user
% documentation in PDF and other supporting files.  File
% \ngefile{nge.dtx} will contain the \LaTeX{} code of all class and
% style files in the package and file \ngefile{nge.ins} will be the
% installation script, which extracts all the \LaTeX{} files from
% \ngefile{nge.dtx} and installs them in the appropriate places of
% \TeX{}\ Live distribution.  File \ngefile{nge.dtx} must also provide
% sources for \nged{} user manual and possibly other documents related
% to the package.
% 
% \nged{} development will follow the principle of creative lazyness:
% no reinvention of wheels, if some task is done well by another
% \TeX{}\ Live package, this task should be delegated to it.
% Duplication of text will also be avoided as much as possible: the
% best way to keep two blocks of text or code identical is to insert
% the same source twice.  Finally, the best way to work rapidly is to
% think slowly and provide plenty of explanations.  The package will
% have documented as much as possible, even its own documentation.
%
% Thus \nged{} will be implemented as follows.  All the \LaTeX{}
% sources together with their documentation and sources of all the
% supporting files for distribution package will be contained in the
% single source file \ngefile{nge.dtx}.  File \ngefile{nge.dtx} will
% be structured so that all the parts of the distribution package for
% upload on CTAN, including the installation script \ngefile{nge.ins}
% will be extracted from \ngefile{nge.dtx} with the help of
% \docstrip{} utility.  File \ngefile{nge.dtx} will also document its
% own structure.  Thus \ngefile{nge.dtx} is the only source file of
% \nged{} distribution package.
% 
% The generation and development of \nged{} package will be aided by a
% make file \ngefile{Makefile}, which will be evolved in parallel with
% \nged{}.  This makefile will not be a part of \nged{} distribution
% package, but it will be a part of \nged{} source package.  Its
% development will be also documented in appropriate detail in
% \ngefile{nge.dtx}.  Thus \ngefile{nge.dtx} and \ngefile{Makefile}
% are the only two files of \emph{\nged{} source package}.  These two
% files together with \LaTeX{} packages, required for application of
% all \nged{} class and style files and generation of all \nged{}
% documentation form \emph{\nged{} development package}.
%
% \nged{} must provide three types of documentation: end user manual,
% which discusses how to use \LaTeX{} package, administrator manual,
% which deals with installation and configuration questions, and
% developers manual, which discusses the maintenance and further
% development of the package.  In addition, maintainers of \nged{} may
% have their private documentation, which, for obvious reasons, is not
% discussed here.
% 
% Documentation of \nged{} must provide for several entry points into
% the package: a user, trying to decide whether the package could be
% useful, may begin with reading the manual, available as a PDF file
% on CTAN.  A system administrator may start with reading the file
% \ngefile{README.txt} in the distribution package.  A \LaTeX{}
% programmer may start reading the source file \ngefile{nge.dtx} directly.  In
% all these cases clear and concise instructions must be placed near
% the beginning telling where to find or how to build the complete
% documentation.  Chains of references must be short and their length
% should not exceed one in most of the cases and must never exceed
% three.  It is better to tolerate some duplication of documentation,
% than to force the user to chase one reference after another.
% 
% \nged{} development package will be placed under version control in
% a Subversion repository at \ngedsvnroot.
%
% Other design decisions pertinent to particular features of \nged{}
% will be made below as such a necessity arises.
%
% \subsection{Main source file: \ngefile{nge.dtx}}
% \label{sec:dev-dtx}
%
% We begin with empty file \ngefile{nge.dtx} and introduce its structure step
% by step adding more and more details.  Contrary to the common
% approach, when \LaTeX{} code is developed by introducing more and more
% parts of it which are finally extracted into a class or style file,
% the exposition in this section is build around various pieces of
% \emph{quoted} code, which were extracted from other parts of this
% file.
%
% Let us summarize the requirements for \ngefile{nge.dtx} formulated in
% section \ref{sec:dev-design}.  File \ngefile{nge.dtx} contains \LaTeX{} code
% for all \nged{} class and style files together with their
% documentation.  It also contains sources of all the supporting files
% of \nged{} distribution package and their documentation.  And
% finally, it contains its own documentation too, part of which is
% provided by this section.  We shall adopt the following structure of
% \ngefile{nge.dtx}.  The file will consist of two parts: head and body
%\iffalse
%<*hide>
%\fi
% \begin{Verbatim}[frame=single,commandchars=\\\{\}]
\meta{head}
\meta{body}
% \end{Verbatim}
%\iffalse
%</hide>
%\fi
% Part \meta{head} contains the sources of all the supporting files,
% while part \meta{body} contains all the \LaTeX{} sources for class
% and style files, their documentation and the package documentation
% too.
%
% In order to combine \LaTeX{} code and its documentation in one file
% we shall use \LaTeX{} package
% \latexpackage{doc}\footnote{\texcatalogueentry{doc}}.  The \LaTeX{}
% code for \nged{} class and style files in file \ngefile{nge.dtx} is marked
% up according to \latexpackage{doc} conventions and is interpspersed
% with comments, which serve as \LaTeX{} source for documentation.
% Actual \nged{} class and style files are extracted from \ngefile{nge.dtx} by
% \docstrip{}\footnote{\texcatalogueentry{docstrip}} utility and the
% package documentation produced by processing \ngefile{nge.dtx} with
% \LaTeX{}.  Thus \ngefile{nge.dtx} has the form
%\iffalse
%<*hide>
%\fi
% \begin{Verbatim}[frame=single,commandchars=\\\{\}]
\textit{sources for supporting files in \nged{} distribution package}
\textit{\latexpackage{doc} configuration}
\textit{latex code for classes and styles code with \docstrip{} markup}
  \textit{and their documentation}
\textit{\nged{} documentation source}
% \end{Verbatim}
%\iffalse
%</hide>
%\fi
% The detailed structure of each part is discussed in subsections
% below.
%
% \subsubsection{The driver}
% \label{sec:dev-dtx-driver}
%
% By convention, the part of file \ngefile{nge.dtx}, containing
% \latexpackage{doc} configuration is called the \emph{driver}.  We
% shall use in the driver \LaTeX{} class \latexclass{ltxdoc}, which
% loads \latexpackage{doc} automatically.  Thus initially the driver
% looks as follows
% \VerbatimInput{driver-skel.tex}
% The lines between |\begin{document}| and |\end{document}| contain
% commands for \latexpackage{doc} to read and process files with the
% code and documentation.  In our case there is only one file ---
% \ngefile{nge.dtx} itself and we use command
% \VerbatimInput{driver-body}
% But running |latex| on file
% \VerbatimInput{driver-2.dtx}
% will produce an error, since after reading line
% \VerbatimInput{driver-body}
% \LaTeX{} will begin reading \ngefile{nge.dtx} from the beginning and during
% the second processing of this file \LaTeX{} again sees
% |\documentclass{ltxdoc}|.  In order to avoid this error everything
% between |\documentclass{ltxdoc}| and |\end{document}| in \ngefile{nge.dtx}
% needs to be concealed from \LaTeX{} on the second reading.  
%
% In ordinary \TeX{} one could conceal parts of a source file from
% reading by ``commenting out''.  If an unquoted \% sign occurs in a
% line of a tex source file the first occurence of this sign together
% with everything following it till the end of the line is considered
% a comment and is skipped by \TeX{} during the processing.
% Alternatively, we can tell \TeX{} to skip a block of lines by
% placing it inside a conditional statement whose condition is never
% fulfilled, that is between |\iffalse| and |\fi| commands.  \LaTeX{}
% follows the same conventions as \TeX{}.  But commenting out the
% entire driver or placing it between |\iffalse| and |\fi| creates
% another problem.  Lines
% \VerbatimInput{driver-3.dtx}
% are concealed from \LaTeX{} on the first run too and they are never
% seen by \LaTeX{} at all.
%
% In order to make the driver visible to \LaTeX{} on the first run and
% invisible on the second run, when the command |\DocInput{\jobname}|
% is processed, we can use the fact, that package \latexpackage{doc}
% changes the meaning of character \%.  With \latexpackage{doc} loaded
% \% no longer indicates the beginning of a comment, it is treated by
% \LaTeX{} in the same way a blank space is treated.  The driver
% \VerbatimInput{driver-4.dtx}
% is processed by \LaTeX{} correctly and produces no errors: it is
% seen by \LaTeX{} on the initial pass over \ngefile{nge.dtx} and everything
% between |%\iffalse| and |%\fi| is skipped on the second pass.
%
% \subsubsection{Use of comments}
% \label{sec:dev-dtx-comments}
%
% In the section \ref{sec:dev-dtx-driver} we concealed the driver of
% \ngefile{nge.dtx} from \latexpackage{doc} by placing it between commands
% |%\iffalse| and |%\fi|. We shall use this method of commenting
% systematically during the preparation of \ngefile{nge.dtx}.  Comments in
% \ngefile{nge.dtx} are used mostly to conceal various parts of this file from
% \LaTeX{} and not to provide detailed information about the file
% structure.  The latter purpose is served by the document you are
% currently reading.
%
% Every part of \ngefile{nge.dtx} which needs to be concealed from \LaTeX{}
% will be either commented out with \% signs or placed between
% |\iffalse| and |\fi| commands.  Similarly, every part which needs to
% be concealed from \latexpackage{doc} package during the stage, when
% will be placed between |%\iffalse| and |%\fi| commands.
% With the help of these techniques it is possible to achieve all four
% possible combinations of visibility and invisibility of a particular
% part of \ngefile{nge.dtx} to \LaTeX{} on the first pass over \ngefile{nge.dtx} and
% \LaTeX{} on the second pass with \latexpackage{doc} already
% loaded.
%
% Beware, however that using |\iffalse| and |\fi| or |%\iffalse| and
% |%\fi| for commenting purposes requires care, since \TeX{} still reads
% everything following |\iffalse| in search of a matching |\fi|.  This
% may lead to unexpected results, if \TeX, for example, encounters a
% construct |\iffalse...\fi...\fi|.  The first |\fi| is not ignored,
% instead it matches the preceding |\iffalse|, thus ending a comment.
% When \TeX{} encounters the second |\fi|, this may result in an
% error, if this |\fi| does not match any prior |\iffalse| or the
% second |\fi| may match some prior |\iffalse| for which it was not
% intended.
% 
% In addition \latexpackage{doc} recognises sequence of characters
% |^^A| as the beginning of a comment and skips everything, following
% the first occurence of this unquoted character string till the end
% of the line.  This string has the same relation to the pair
% |%\iffalse| and |%\fi| in \latexpackage{doc} package, as character
% |%| to the pair |\iffalse| and |\fi|.  We conceal from
% \latexpackage{doc} those parts of \ngefile{nge.dtx} which provide sources
% for other files in \nged{} distribution package with the help of
% |%\iffalse| and |%\fi|, and use |^^A| for short comments in
% \ngefile{nge.dtx}, helping to separate one part of the file from another and
% serving as reminders of various conventions adopted during the
% development.
% 
% \subsubsection{Configuration of \latexpackage{doc} package}
% \label{sec:dev-dtx-doc}
% 
% After the line |\documentclass{ltxdoc}| we add configuration
% commands for \latexpackage{doc}
% \VerbatimInput{driver-config.tex}
% which turn on indexing of macros and refer to line numbers of code
% where they were used.  We also tell \latexpackage{doc} to record
% history of changes, made in the file.  These configuration commands
% for \latexpackage{doc} are also supplemented by the command
% \VerbatimInput{stopeventually.dtx}
% telling \LaTeX{} to perform appropriate actions at the end of
% typesetting.  Configuration of \latexpackage{doc} used in processing
% \ngefile{nge.dtx} may by changed by the user with the help of 
% configuration file \latexfile{ltxdoc.cfg}.  For details see
% section \ref{sec:user-manual}.
%
% Later we shall add to \latexpackage{doc} configuration commands in
% the preamble of the driver other commands, loading various \LaTeX{}
% packages, configuring them, and defining custom macros.
%
% Right after the driver we also place commands
% \VerbatimInput{checksum.dtx}
% providing rudimentary error control for \ngefile{nge.dtx} during network
% transmissions.  Thus the initial version of \ngefile{nge.dtx} takes the form
% \VerbatimInput{nge-1.dtx}
% User documentation will go before |\StopEventually| command and the
% code and package documentation will follow this command.
%
% \subsubsection{\docstrip{} markup}
% \label{sec:dev-dtx-docstrip-markup}
%
% \docstrip{} is used in \nged{} for two purposes.  All the class and
% style files for \nged{} are extracted from \ngefile{nge.dtx} with the help
% of \docstrip{} utility.  Other parts of \ngefile{nge.dtx} are extracted with
% the help of \docstrip{} too.  Some of these extracted parts become
% components of \nged{} distribution package, some are inserted
% verbatim in the package documentation as quotations.  The latter
% solves the consistency problem: it ensures that the quotation in the
% documentation is indeed identical with the part of the code to which
% it is referring.
%
% The extraction of parts is done with the help of \docstrip{}
% |\generate| command.  The details are given in sections
% \ref{sec:dev-dtx-docstrip-ins},
% \ref{sec:dev-dtx-docstrip-extraction}.  In this section we shall
% discuss the structure of \ngefile{nge.dtx} from the point of view of
% \docstrip.
%
% In its default configuration \docstrip{}, when run on a \TeX{} file,
% removes all the lines, which begin with a single |%| sign and
% outputs other lines literally.  There is also an important
% exception: lines beginning with |%%| are not stripped, they are also
% output literally.  For other details see \docstrip{}
% documentation\fixme[reference].
%
% In order to simplify generation of multiple files from a single
% source, \docstrip{} makes possible conditional extraction of lines.
% This is done by giving |\generate| command an options list with with
% which the input file is processed 
%\iffalse
%<*hide>
%\fi
% \begin{Verbatim}[commandchars=\+\[\]]
\generate{\file{+textit[file.out]}{\from{+textit[file.in]}{+textit[opt1,+dots,optn]}}} 
% \end{Verbatim}
%\iffalse
%</hide>
%\fi Every line of \latexfile{file.in} is also supplied with a boolean
% expression of options in disjunctive normal form.  Such expressions
% are called by \docstrip{} authors \emph{guards}.  One may view
% quards as boolean functions which depend on boolean variables
% \textit{opt1}, \dots, \textit{optn} and may depend on other boolean
% variables (options) too.  The list of options, given to |\generate|
% command tells, which variables in \meta{guard} evaluate to |true|.
% All other variables in the guard evaluate to |false|.  Then
% \latexfile{file.in} is read line by line and the extraction decision
% is made according to the following criteria\footnote{Warning: these
%   conclusions are based on my understanding and interpretation of
%   \docstrip{} manual.  I have not studied \docstrip{} code itself.
%   In cases where there is any doubt about the accuracy of this
%   description, it is strongly advised to try experiment with
%   \docstrip{}.}.
% \begin{enumerate}
% \item A line is a \emph{comment line} if it begins with |%| which is
%   not followed by a guard or another |%|.  All the comment lines are
%   stripped.
% \item If the line begins with a guard \meta{guard}, possibly
%   prefixed with |%|, then boolean expression \meta{guard} is
%   evaluated according to the standard rules of boolean logic.  All
%   the boolean variables from the list of options \textit{opt1},
%   \dots, \textit{optn} evaluate to |true|, any other boolean
%   variables, occuring in \meta{guard} evaluate to |false|.  If
%   boolean function $\meta{guard}$ evaluates to |true|, then
%   everything following the guard to the end of the line is
%   extracted.  Otherwise the line is skipped.
% \item If the line is not a comment line and it has no guard, it is
%   extracted unconditionally, that is \emph{always}
%   extracted\footnote{This criterion is strikingly different from the
%     usual treatment of disjunctive normal forms (DNFs) in
%     mathematical logic, where an empty DNF always evaluates to
%     |false|.  Interpreting the empty DNF as a constant boolean
%     function with value |false| would mean that failing to provide a
%     guard for a line would make this line effectively unextractable.
%     Naturally, no matter how consistent it is with DNFs' notations,
%     this is not the behaviour one expects from an extraction
%     process.  Even when no options for extraction are specified we
%     expect the line to be extracted unless it is a comment line.
%     But it is precisely this deviation from the standard
%     interpretation of DNFs, which makes creation of extraction rules
%     harder than necessary.  A different set of defaults is forced
%     upon the language of DNFs, which make me wonder, whether the
%     initial choice of the language for boolean expressions was well
%     thought through.  Conjunctive normal forms (CNFs) look like a
%     more suitable alternative here, but dealing with CNFs is much
%     harder for humans than dealing with DNFs.  Is there a better
%     form of presenting options for extractions, which makes it easy
%     to comprehend by humans and consistent with set-theoretic
%     operations?}.
% \item A line which begins with |%%| is considered a comment line,
%   marked for extraction.  It is processed for extraction in the same
%   way as a line with an empty guard: it is extracted entirely,
%   together with the leading |%%|.
% \end{enumerate}
% If a block of contiguous lines has to be marked for extraction with
% the same guard \meta{guard}, instead of marking each line with
% \meta{guard}, one can enclose the entire block between two anchors
% \meta{*guard} and \meta{/guard}, produced from \meta{guard} with
% modifiers |*| and |/|.  Guard \meta{guard} with guard modifier |*|
% marks the beginning of the block, marked with \meta{guard}, and
% \meta{guard} with guard modifier |/| marks the end of this block.
% Simple guards may be nested.  Constructs like
%\iffalse
%<*hide>
%\fi
% \begin{Verbatim}
<*A>
<B> some text   
</A>
% \end{Verbatim}
%\iffalse
%</hide>
%\fi
% and
%\iffalse
%<*hide>
%\fi
% \begin{Verbatim}
<*A>
<*B>
 some text
</B>   
</A>
% \end{Verbatim}
%\iffalse
%</hide>
%\fi
% are equivalent to 
%\iffalse
%<*hide>
%\fi
% \begin{Verbatim}
<A&B> some text
% \end{Verbatim}
%\iffalse
%</hide>
%\fi
% provided that expression |<A&B>| is well defined for \docstrip{}.
%
% While preparing \ngefile{nge.dtx} we adopt the following conventions.  In
% order to avoid inadvertent extraction of lines into multiple files,
% \emph{all} \LaTeX{} code will be marked up with guards.  Most of the
% quards will have mnemonic names, clearly relating them to the part
% of \ngefile{nge.dtx} they are guarding.  Here is, for example, the driver of
% \ngefile{nge.dtx} with appropriate guards incorporated
% \VerbatimInput{driver-template.dtx}
% Note that quards |<*driver>| and |</driver>| are not seen by
% \LaTeX{}, since they are prepended by the comment sign |%|.  These
% guards are not seen by \latexpackage{doc} package either, since the
% entire driver is placed between |%\iffalse| and |%\fi|.
%
% Concealing guards from \latexpackage{doc} may require additional
% care.  For example, the line
%\iffalse
%<*hide>
%\fi
% \begin{Verbatim}
%<guard> some text   
% \end{Verbatim}
%\iffalse
%</hide>
%\fi 
% will be processed by \docstrip{} correctly: if option |<guard>| is
% given, | some text| will be output, and the line will be ignored
% otherwise.  Similarly, this line will be ignored by \LaTeX{},
% because it begins with a comment character |%|.  But
% \latexpackage{doc} will see this line and it will be typeset as\\
% \leftline{<guard> some text}
% which presumably was not our intention.  In order to conceal the
% guard from \LaTeX{} and \latexpackage{doc} it has to be placed
% between |%\iffalse| and |%\fi| commands.  Hence the correct way to
% prepare the line above for typesetting by \LaTeX{} is
%\iffalse
%<*hide>
%\fi
% \begin{Verbatim}
%\iffalse
%<*guard>
%\fi
 some text   
%\iffalse
%</guard>
%\fi 
% \end{Verbatim}
%\iffalse
%</hide>
%\fi 
% If we want to conceal this line from \LaTeX{} and make it visible to
% \latexpackage{doc} only, we can prepend | some text| with |%|.
% \iffalse
%<*hide>
%\fi
% \begin{Verbatim}
%\iffalse
%<*guard>
%\fi
% some text   
%\iffalse
%</guard>
%\fi 
% \end{Verbatim}
%\iffalse
%</hide>
%\fi 
% This does typeset the line above correctly, but produces another
% problem: even with option |guard| given explicitely \docstrip{} will
% view  line |% some text| as a comment line and it will strip it off.
% If we still want this line to be in \docstrip{} output, we have to
% prepend it not with one |%|, but with |%%|.  
%\iffalse
%<*hide>
%\fi
% \begin{Verbatim}
%\iffalse
%<*guard>
%\fi
%% some text   
%\iffalse
%</guard>
%\fi 
% \end{Verbatim}
%\iffalse
%</hide>
%\fi 
% This way, as explained above, \docstrip{}\footnote{provided that
%   docstrip is in its default configuration} will consider line
% |%% some text| as a comment,
% intended for extraction and it will be extracted from the file.
% That is why the lines for |\CheckSum|, |\CharacterTable|,
% |\StopEventually| and |\Finale| in the examples in section
% \ref{sec:dev-dtx-doc} were prepended not with one, but with two |%|
% signs.  This way they are processed correctly by both
% \latexpackage{doc} package and \docstrip{}.  
%
% Guard |<hide>| will be used for marking up parts of \ngefile{nge.dtx} which
% are never to be extracted with \docstrip{}, with the explicit
% assumption, that option |hide| will \emph{never} be given to
% |\generate| command.  Guard |<hide>| is used in particular to
% conceal lines containing |<guard>| in examples above.  Here is how
% the source for the first of the three examples above looks like
% \iffalse
%<hide-example>%\iffalse
%<hide-example>%<*hide>
%<hide-example>%\fi
%<hide-example>% \begin{Verbatim}
%<hide-example>\iffalse
%<hide-example><*guard>
%<hide-example>\fi
%<hide-example> some text   
%<hide-example>\iffalse
%<hide-example></guard>
%<hide-example>\fi 
%<hide-example>% \end{Verbatim}
%<hide-example>%\iffalse
%<hide-example>%</hide>
%<hide-example>%\fi 
% \fi
% \VerbatimInput{hide-example.dtx}
% There is also a simple way to place in docstrip output a line which
% begins with exactly one |%| character, provided, that this line does
% not need to be processed by \latexpackage{doc}.  For this it
% suffices to place the line between |%\iffalse| and |%\fi| commands and
%  prepend the entire line with an appropriate guard.
%\iffalse
%<*hide>
%\fi
% \begin{Verbatim}
%\iffalse
%<guard>% some text   
%\fi 
% \end{Verbatim}
%\iffalse
%</hide>
%\fi
% After running \docstrip{} with option |guard| on the source, line\\
% |% some text|\\
% will be extracted into another file, whose contents may be then
% quoted verbatim with the help of |\VerbatimInput| command.
% This is how the example above with a hidden |<guard>| was
% produced.
%
% The rest of the file \ngefile{nge.dtx} consists of parts, marked up with the
% help of the techniques extensively discussed above.  In particular,
% every subsection of section \ref{sec:code} is enclosed between a
% pair of anchors, whose guard name is derived from the appropriate
% class or style file, derived in this section.  For example, entire
% subsection \ref{sec:code-ngemin} is enclosed between anchors
% |<*ngemin>| and |</ngemin>|.  Similarly, the part of \ngefile{nge.dtx} file,
% which serves as the source for readme.txt file of \nged{}
% distribution package is placed between anchors |<*readme>| and
% |</readme>|, and so on\dots.  In addition to guard |<hide>|, which
% is never issued as option to \docstrip{}, we also use several
% guards, which are common for some groups of generated files.  For
% example, guard |<code>| marks up the part, which is common for all
% the \LaTeX{} class and style files.  It includes in particular
% version information for each file.
%
% \subsubsection{Extracting parts of itself}
% \label{sec:dev-dtx-docstrip-extraction}
%
% At the end of section \ref{sec:dev-dtx-docstrip-markup} we have
% discussed, how various pieces of \ngefile{nge.dtx} may be marked up for
% extraction with \docstrip{}.  Some of the extracted pieces are later
% inserted in the documentation verbatim, next to the explanations of
% their structure and purpose.  Other extracted pieces of \ngefile{nge.dtx}
% become part of \nged{} distribution package.
%
% We place \TeX{} commands, controlling the extraction, in the very
% beginning of \ngefile{nge.dtx}.  These are the first lines \LaTeX{} sees and
% processes even before seeing |\documentclass| command.  This way all
% the parts of \ngefile{nge.dtx} are extracted before they are needed for
% quoting at later stages of processing.
%
% 
% The extraction is done with the help of \docstrip{} utility, so we
% load it first.
% \VerbatimInput{ext-docstrip.dtx}
% Then we configure \docstrip{} to extract files silently, overwriting them if
% they necessary and tell it to prepend or  append nothing  to
% extracted pieces.
% \VerbatimInput{ext-mock-config.dtx}
% In order to conceal the extraction commands from \latexpackage{doc}
% package, we need to place them between |%\iffalse| and |%\fi|
% commands.  So the beginning of \ngefile{nge.dtx} now takes the form
% \VerbatimInput{ext-skel.dtx}
% Now it remains only to add file generation commands.  The code below
% extracts pieces of \ngefile{nge.dtx} for quoting them later in the
% documentation.
% \VerbatimInput{ext-examples.dtx}
% File \latexfile{canary.txt} checks for lines with \LaTeX{} commands
% without \docstrip{} guards.  If there are no such lines it will be
% empty.
%
% In the same way one can add commands for generation other files in
% \nged{} distribution package.
%
% \subsubsection{File \ngefile{nge.ins}}
% \label{sec:dev-dtx-docstrip-ins}
%
% File \ngefile{nge.ins} is part of \nged{} distribution package.
% Its purpose is to generate all class and style files from
% \ngefile{nge.dtx}.  Its structure is similar to the structure of code at the
% end of section \ref{sec:dev-dtx-docstrip-extraction}.  But there
% docstrip has different configuration in it. First of all\fixme[complete]
% \VerbatimInput{ins-1.dtx}
% Furthermore \fixme[complete]
% \VerbatimInput{ins-texlive.dtx}
% In addition, all the class and style files have license and
% copyright notice inserted in them during the generation process.
% This is achieved by supplying appropriate preamble in
% \ngefile{nge.ins}:
% \VerbatimInput{ins-preamble.dtx}
% Then file generation commands are issued
% \VerbatimInput{ins-body.dtx}
% followed by installation message and ``end of job'' command
% \VerbatimInput{ins-tail.dtx}
% 
% \subsubsection{Extracting everything from \ngefile{nge.dtx}}
% \label{sec:dev-dtx-docstrip-all}
%
% Since the structure of \ngefile{nge.ins} is so similar to the
% extraction code in the beginning of \ngefile{nge.dtx}, quoted in section
% \ref{sec:dev-dtx-docstrip-extraction}, it is tempting to avoid
% duplication of efforts and arrange file extraction commands in the
% beginning of \ngefile{nge.dtx} so that file \ngefile{nge.ins} is generated
% automatically, with its preamble extracted from the part of
% \ngefile{nge.dtx}, which contains the text of copyright notice and license.
%
% This is achieved by interspersing \docstrip{} configuration and
% extraction commands in the beginning of \ngefile{nge.dtx} with the text of
% copyright notice and license, and marking them up appropriately with
% various guards.  It is also possible to obtain the same result by
% keeping docstrip commands and license text in separate parts of
% \ngefile{nge.dtx}, and generating \ngefile{nge.ins} with the help of several
% |\from| commands.  We do not use this approach for reasons which
% will be given below.
%
% First we begin by placing all the \docstrip{} configuration end extraction
% commands  inside a \TeX{} group, in order to avoid interference of
% \docstrip{} definitions with \LaTeX{} definitions, loaded later.
% \VerbatimInput{ext-head.dtx}
% Then we load docstrip and its configuration commands.
% \VerbatimInput{ext-docstrip.dtx}
% Since configuration commands for file extraction from \ngefile{nge.dtx} and
% \ngefile{nge.ins} differ from each other, some of them need to be
% concealed from \docstrip{}, while others have to be concealed from
% \LaTeX{}.  This is achieved with the help of |<ins>| guard, marking
% parts for extraction into \ngefile{nge.ins}.
% \VerbatimInput{ext-fullconfig.dtx}
% In the same way we configure the preamble.  Different preambles a
% needed for generation of class and style files, for package
% distribution files and for code examples.  Therefore, next we place
% in \ngefile{nge.dtx} the text of license, marked up for extraction into
% preamble of \ngefile{nge.ins} and also into a separate file.
% \VerbatimInput{ext-preamble.dtx}
% Next we give file generation commands\footnote{File generation
%   commands quoted here correspond to the development version of
%   \ngefile{nge.dtx}.  In the version of \ngefile{nge.dtx},
%   distributed with \nged{} package some of these commands are
%   commented out in order to avoid overwriting files, distributed
%   with the package.}.
% \VerbatimInput{ext-body.dtx}
% Note that in the code above file generation commands for class and
% style files come first.  Therefore the text of the license given in
% preamble to these command applies to them.  The same preamble is
% also extracted into \ngefile{nge.ins}, which makes class and style
% files, with the help of lines above, identical with class and style
% files, extracted from \ngefile{nge.dtx} with the help of file
% \ngefile{nge.ins} given in section
% \ref{sec:dev-dtx-docstrip-ins}. Furthermore, the same preamble
% applies to file \ngefile{nge.ins} itself, extracted from
% \ngefile{nge.dtx} by the code above.  It is precisely this ability
% to use the same text of license several times which affected our
% decision to intertwine \docstrip{} commands with license text in
% \ngefile{nge.dtx}.
%
% Extraction commands are followed by the rest of \ngefile{nge.ins}
% contents, after which we close the \TeX{} group.
% \VerbatimInput{ext-tail.dtx}
% The |\ifx| conditional above checks whether accidentally or
% intentionally the file generation commands were processed with plain
% \TeX{} and if this is the case, exits.  Otherwise, the \TeX{} group,
% opened in the beginning of this section, is closed and the program
% (presumably \LaTeX{} or its relative) proceeds to read and interpret
% the driver.
%
% \subsection{Configuring packages}
% \label{sec:dev-dtx-packages}
%
% \subsubsection{Subversion support}
% \label{sec:dev-dtx-packages-svn}
%
% We use \latexpackage{svninfo}.  Here is the information available to
% us:\\ \svnId\\
% It is accessible through the following commands\\
% |\svnInfoFile| \svnInfoFile\\
% |\svnInfoRevision| \svnInfoRevision\\
% |\svnInfoDate| \svnInfoDate\\
% |\svnInfoTime| \svnInfoTime\\
% |\svnInfoOwner| \svnInfoOwner\\
% |\svnInfoYear| \svnInfoYear\\
% |\svnInfoMonth| \svnInfoMonth\\
% |\svnInfoDay| \svnInfoDay\\
% |\svnInfoLongDate| \svnInfoLongDate\\
% |\svnToday| \svnToday
%
% \section{Misc unprocessed}
% \label{sec:misc}
%
% The rest of this section contains various bits and ends: ideas and
% things to try, notes to self and other developers, lists of bugs,
% etc.
%
% \subsection{Bugs}
% \label{sec:dev-bugs}
%
% \subsection{Bits and ends}
% \label{sec:dev-misc}
% To be processed.
% \begin{list}{}{Things to do and to think about}
% \item Configure todos in this file
% \item Use svn support too for tags and in latex
% \item Think through font selection for package names, file names and
%   commands issued.
% \item Find a way to incorporate ChangeLog file.
% \item Add style for url references.
% \item Write about mirroring manual parts for user and developer.
% \item Add paragraph about making releases.
% \item When nge.dtx is processed through latex, issue installation
%   instructions in the end, not at the beginning.
% \item Add possibility to make todo and fixme notes right in the text
% \item Mention in the manual packages which \nged{}uses and what is
%   needed to compile documentation.
% \item update message and sync it with installation instructions.
% \item sync file list with license with what is generated by the
%   package
% \item make a list in the document of all the files, generated by it.
% \item test nge.ins with local docstrip.cfg.
% \item Think through typographic conventions: urls, package names,
%   directories, files, what else?
% \item Combining several documents (exams, quizzes, etc\dots) in one
%   document. Use report class?  \textsf{Doc} package and docstrip?
% \item ``postmortem'' command and environment.
% \item Is there a simple way to keep an up to date manifest.txt?
% \item the package should not depend on files outside \TeX{}\ Live
%   distribution.
% \item fix \verb|\showdirectory| call in \meta{ins} section.
% \item Two possibilities --- all problems back to back, or each
%   problem on a new page.  in the latter case an option to provide a
%   space for the answer and option to provide place for a problem
%   grade.
% \item Compile list of problems with totals points in a table.
%   options for placing this table on the front page, standalone table
%   on the second page or on the last page.
% \item multiple variants.  options for solutions, rubrics.  Notes to
%   students and to graders. Assesment notes.
% \item compilation of all course materials under the same title,
%   similar to doc package.
% \item configuration file, where a common theme may be configured.
% \item Rudimentary control over vertical space for problems.
% \item Localization.
% \item Types of problems: multiple choice (circle, underline answer,
%   fill in the blanks), true or false, problems with parts (one
%   level).
% \item examdesign produces multiple (randomized?) versions
% \item Simultaneous generation of exams (with versions), solutions
%   and rubrics in one batch, but in separate files.
% \item most options (where it makes sense) should provide a
%   possibility of manually overriding them, at least in principle.
% \item Multiple instructors, multiple sections -- think what to do in
%   that case.
% \item Write about typographic conventions for the exam (e.g.
%   display style formulas vs. inline style formulas).
% \item in nyu classes may be it is possible to generate pages for
%   scantrone?
% \item partial grading tables?  Not sure whether it is a good idea
% \item use hyperref in PDF mode
% \item provide for bonus problems and points
% \item write a manual with plenty of examples
% \item I think automatic generation of dtx file for release is
%   unavoidable.  But still I would like to break it apart as little
%   as possible.  May be it is possible to keep everything in it and
%   process it conditionally depending on whether it is a release
%   version or a developlment version.  For now I am not going to
%   worry about that.
% \item Write a simple script which guards agains common typos
%   (forgotten braces after macros like \verb|\LaTeX|, repeated words,
%   etc.  Perhaps it should be written in perl.
% \item In addition to system-wide configuration there must be a
%   possibility to configure \nged{} in a single directory, so that it
%   processes all the sources using the common theme, but which may be
%   overriden in individual documents.
% \item Review nged.dtx.  Separate cleanly style and logic.  Must have
%   default behaviour similar to article.cls
% \item Sensible sets of defaults?
% \item Distinguish release package, release tools, release sources,
%   development sources, development tools and personal development
%   environment and structure Makefile accordingly.
% \item Write a short manual for developing too
% \item perhaps it is better to use group environment for sections
%   included conditionally.
% \item Try a docstrip.cfg, so that it is as silent as possible during
%   development
% \item look at svn support in latex and emacs
% \item Add discussion about file structure
% \item write introduction about nged in broad view
% \item produce reports and assessments in doc style?
% \item Is is possible to make releases with docstrip?
% \item 2 rolling released branches --- stable and dev. also my
%   private development branch kdv.
% \item title pages are just special pages
% \item course name macro, what shall it include?
% \item course info: title, catalog number, timestamp (i.e. academic
%   year, semester, etc, think of a better term in here).
% \item If something is omitted, such as course title, author, etc, it
%   should not be printed, and must not produce error.  The only
%   exception is the title and \verb|maketitle| command.
% \item pdf output may provide something which is not available in dvi
%   format.
% \item when releasing package update checksum
% \item Do we need a bibliography with CTAN URLs for packages?
% \item figure out a way to include draft options conditionally
% \item option for grade table on the back of the last page
% \item two modes for problems: ``exam'' mode and review mode (turn
%   on/off grade totaling, provide or not provide space for marks,
%   check of the problem has several parts on one line, so that there
%   is no clutter, a column of grades for parts (even a little table).
% \item files nge.cfg, ngecourse.def (or courseinfo.def) with sensible
%   defaults, (also ngetheme.cfg?)
% \item Do I need \latexpackage{url} if I already use
%   \latexpackage{hyperref}?
%  \end{list}
%
% Here is the list of todo notes located throughout the text.
% \listoftodos 
% \paragraph{Courses}
% \nged{} is built around one logical entity -- a course.  All the
% documents are unified under course theme.  It is also possible to
% extend this to course sequence, but most of the time it's just one
% course.
% 
% Each course usually has the following metadata associated with it
% \begin{list}{}{course metadata}
% \item Course name (also a short name)
% \item Educational institution
% \item Educational term and year (semester, quarter, trimester, period\dots)
% \item Instructor or multiple instructors with sections.  Possible
% recitations, precepts, problem sessions, labs, run by teaching
% assistants.  There may be graders too.  There may be course
% coordinator or some other administrative figure.  Do we really need
% to provide for all this information?  Not clear.
% \end{list}
% Course documents may include tests (quizzes, exams\dots),
% assignments (homeworks, projects\dots), handouts with various
% information (syllabus, schedules, requirements, \dots). There also
% may be problem solutions, various postmortems, section and
% instructor assesments, course assesments.  All of these documents may
% have titles, dates and authors as well.
%
%^^A \IfFileExists{ChangeLog}
%^^A {\subsection{ChangeLog entries}\label{sec:dev-changelog}
%^^A \VerbatimInput{ChangeLog}}
%^^A {\relax}
%
%\iffalse
%^^A The next line is added for formatting purposes
%<nge-1>%
%<*finale>
%\fi
%% \Finale
%\iffalse
%</finale>
%\fi
%
%^^A Local Variables:
%^^A TeX-master: t
%^^A End:
