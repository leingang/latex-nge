% \iffalse (This is a meta-comment)
% $Id$
%
% Copyright (C) 2003--2009 Matthew Leingang <leingang@courant.nyu.edu>
% --------------------------------------------------------------------
% 
% This file may be distributed and/or modified under the
% conditions of the LaTeX Project Public License, either version 1.2
% of this license or (at your option) any later version.
% The latest version of this license is in:
%
%    http://www.latex-project.org/lppl.txt
%
% and version 1.2 or later is part of all distributions of LaTeX 
% version 1999/12/01 or later.
%
% \fi
% \iffalse 
\input nged.def
%\NeedsTeXFormat{LaTeX2e}
%<examclass>\def\ngefiledescription{NGED class for math exams}           \ProvidesClass{ngeexam}
%<quizclass>\def\ngefiledescription{NGED class for math quizzes}         \ProvidesClass{ngequiz}
%<worksheetclass>\def\ngefiledescription{NGED class for math worksheets} \ProvidesClass{ngewksht}
%<driver>\def\ngefiledescription{NGED driver}                            \ProvidesFile{ngeexam.drv}
%<example>\def\ngefiledescription{NGED Example exam}                     \ProvidesFile{example.tex}
%<examunit1>\def\ngefiledescription{NGED Unit Test 1}                    \ProvidesFile{test_ngeexam1.tex}
%<examunit2>\def\ngefiledescription{NGED Unit Test 2}                    \ProvidesFile{test_ngeexam2.tex}
%<examunit3>\def\ngefiledescription{NGED Unit Test 3}                    \ProvidesFile{test_ngeexam3.tex}
%<forms>\def\ngefiledescription{NGED package for exam/quiz forms}        \ProvidesPackage{ngeform}
%<problems>\def\ngefiledescription{NGED package for problem structure}   \ProvidesPackage{ngeprob}
%<versions>\def\ngefiledescription{NGED package for versioning}          \ProvidesPackage{ngever}
%<variants>\def\ngefiledescription{NGED package for variants}            \ProvidesPackage{ngevar}
%<modes>\def\ngefiledescription{NGED package for document modes}         \ProvidesPackage{ngemode}
%<misc>\def\ngefiledescription{NGED miscellaneous package}               \ProvidesPackage{ngemisc}
%<*dtx>
\def\ngefiledescription{NGED source file}\ProvidesFile{\jobname.dtx}
%</dtx>
[\ngefiledate\space \ngefileversion\space \ngefiledescription]
%<*driver>
\documentclass{ltxdoc}
\usepackage{url}
\usepackage{units}
\usepackage{hyperref}
\newcommand{\defined}[1]{\textbf{#1}}
\usepackage{todonotes}
\EnableCrossrefs         
\CodelineIndex
\RecordChanges
\usepackage{xspace}
\begin{document}
  \DocInput{\jobname.dtx}
\end{document}
%</driver>
% \fi
%
% \CheckSum{0}
%
% \CharacterTable
%  {Upper-case    \A\B\C\D\E\F\G\H\I\J\K\L\M\N\O\P\Q\R\S\T\U\V\W\X\Y\Z
%   Lower-case    \a\b\c\d\e\f\g\h\i\j\k\l\m\n\o\p\q\r\s\t\u\v\w\x\y\z
%   Digits        \0\1\2\3\4\5\6\7\8\9
%   Exclamation   \!     Double quote  \"     Hash (number) \#
%   Dollar        \$     Percent       \%     Ampersand     \&
%   Acute accent  \'     Left paren    \(     Right paren   \)
%   Asterisk      \*     Plus          \+     Comma         \,
%   Minus         \-     Point         \.     Solidus       \/
%   Colon         \:     Semicolon     \;     Less than     \<
%   Equals        \=     Greater than  \>     Question mark \?
%   Commercial at \@     Left bracket  \[     Backslash     \\
%   Right bracket \]     Circumflex    \^     Underscore    \_
%   Grave accent  \`     Left brace    \{     Vertical bar  \|
%   Right brace   \}     Tilde         \~}
%
% \newcommand\LaTeXClass[1]{\textsf{#1}}
% \newcommand\LaTeXPackage[1]{\textsf{#1}}
% \newcommand\LaTeXOption[1]{\textsf{#1}}
% \newcommand\LaTeXFile[1]{\texttt{#1}}
% \newcommand\LaTeXEnvironment[1]{\texttt{#1}}
% \newcommand\option[1]{\texttt{#1}}
% \newcommand{\truefalse}{\normalfont\texttt{true}/\texttt{false}}
% \newcommand\nged{\textsf{nged}\xspace}
%
% \newenvironment{TODO}{\par\noindent\textbf{TODO:}}{\par}
% \renewenvironment{TODO}{\noindent\marginpar{\hfill{\sffamily\bfseries TODO}}}{\par}
%
% \changes{v0.0.1}{2000}{Initial version}
% \changes{v0.1}{2002/12/12}{Revision}
% \changes{v0.2}{2002/04/10}{Packaged into docstrip}
% \changes{v0.2.2}{2002/04/15}{Added Makefile and README}
% \changes{v0.5}{2006/08/14}{Many changes}
% \changes{v0.5}{2006/08/16}{Added support for document versions}
% \changes{v0.5.1}{2006/11/30}{Added |ifsolutions| macro}
%
% \GetFileInfo{\jobname.dtx}
%
% \DoNotIndex{\newcommand,\newenvironment,\newcounter}
% \DoNotIndex{\begin,\end}
% \DoNotIndex{\(,\[,\],\),\item,\myvector,\mymatrix,
%             \Span,\FiniteSet,\setS}
%
% \title{\nged: A Next Generation Educational Bundle}
% \author{Matthew Leingang \\ leingang@courant.nyu.edu}
%
%
% \maketitle
% \tableofcontents
% 
% \section{Introduction}
%
% The \nged bundle is a set of classes and packages used for
% typesetting educational assessments, particularly mathematics but
% not specific to it.  
%
% \subsection{Goals}
%
% The \nged bundle classes and packages should be
%
% \begin{itemize}
%   \item \emph{beautiful}
%   \item \emph{useful} 
%   \item \emph{flexible} items should be configurable
%   \item \emph{modular} This is why the several different packages for different uses
%   \item \emph{lazy} We will use other packages when they are useful
% \end{itemize}
%
% \subsection{History}
%
% I first wrote the exam class in 2003 and have developed it in stages
% since then.  Other pieces have been developed along the way, too.
% Now (2009) I feel the need to bundle it up and distribute it so
% others I collaborate with can produce similar documents.
%
% \subsection{Other classes which do this kind of thing}
%
% Look at examdesign, mathexam, and exam.
% 
% \section{Exams}
%
% \DescribeMacro{\course\oarg{shortname}\marg{name}} Sets the name of
% the course to be \meta{name} for typsetting in the document title.
% For page headings, a short name may be declared at the same time.
% E.g.,
% \begin{verbatim}
% \course[Math 1a]{Math 1a: Introduction to Calculus}
% \end{verbatim}
% 
% \section{Quizzes}
%
% |\maketitle| produces something more compact to allow for more room on the page.
%
% \section{Worksheets}
%
% Nothing here yet.
%
% \section{Problems and their parts}
%
% \section{Answer forms}
%
%
% When typesetting documents used as assessments in a learning
% environment one often needs to provide a form on which the student
% should indicate their answer.  A box that should be checked, a blank
% in which to write an answer, or a bubble to fill are examples of
% common response forms.  
%
% It also helps to provide at the time of writing the assessment the
% answer to the problem, and to use programmatic switches allowing the
% typesetting of the answers.  If the answer is to be set in a
% different color the task is even more tedious and worthy of
% automation.
%
% The purpose of this package is to collect the author's hacks that
% have been copied-and-pasted from quiz to worksheet to exam and back
% again into a single package with a sane interface and ``correct''
% implementation.
%
% \DescribeMacro{\answerbox\oarg{opts}} Set a box in which the
% student is to put his or answer.  There is an optional list of
% arguments in standard \meta{key}=\meta{val} format but no
% mandatory arguments.  Calling \verb|\answerbox| with no argument
% results in an empty answer box.
%
% \begin{description}
% \item[\option{eol=}\meta{\truefalse}] Put the box at the end of
%   the current line, if there is room.  Otherwise start a new line
%   and put the box at the end of that line.  Default is \texttt{false}.
% \item[\option{width=}\meta{dimen}] width of the box.  Default is \unit[1]{cm}.
% \item[\option{height=}\meta{dimen}] height of the box.  Default is \unit[1]{cm}.
% \item[\option{answer=}\meta{answer}] the answer that should be
%   put in the box for credit.
% \item[\option{showanswer=}\meta{\truefalse}] Put the answer in
%   the box.  This is useful if you want to typeset solutions as a
%   separate file.  The default is \texttt{false}.
% \end{description}
%
% \DescribeMacro{\answerbubble\oarg{opts}\marg{text}} Set a
%   ``bubble'' to be filled in by the student to indicate his or her
%   answer.  The mandatory argument (which could be empty) is the text
%   to place in the bubble, i.e., the answer that would be indicated
%   by filling in the bubble.  There are options as well:
%
% \begin{description}
% \item[\option{filled=}\meta{\truefalse}] Fill in this
%   bubble. Default is \texttt{false}.
% \end{description}
%
% \DescribeMacro{\answerbubblerow\oarg{opts}\marg{choices}}
% Set a row of bubbles, one or more of which the student is to fill in
% to indicate his or her answer. \meta{choices} is a
% comma-separated list of choices for the bubbles.  The options are:
%
% \begin{description}
% \item[\option{eol=}\meta{\truefalse}] Put the box at the end of
%   the current line, if there is room.  Otherwise start a new line
%   and put the box at the end of that line.  Default is \texttt{false}.
% \item[\option{answer=}\meta{answer}] the answer or answers that
%   should be filled in.
% \item[\option{showanswer=}\meta{\truefalse}] Fill in the correct
%   answers.  This is useful if you want to typeset solutions as a
%   separate file.  The default is \texttt{false}.
% \end{description}
%
% \section{Document modes}
%
% By \defined{mode} we mean a production of the assessment intended for a specific audience at a specific time:
% \begin{itemize}
% \item In ``regular'' mode the document produced is the one that students will write their
% answers on (audience: students)
% \item In solutions mode, the answers and solutions are typeset along with the questions (audience: students, after the assessment is completed)
% \item In rubrics mode, the scoring guides are typeset along with the answers and solutions (audience: graders)
% \item In assessment mode, notes about the problems and the students' performance are included along with the answers, solutions, and rubrics (audience: teachers, including older, more experienced forms of the original author)
% \end{itemize}
%
% \section{Document variants}
%
% By \defined{variant} we mean a production of the assessment
% similar to the original, differing only in minor ways to promote academic
% honesty.
%
% \section{Document versions}
%
% By \defined{version} we mean an instance of the master document
% marked by a point in the document's development, as in software development.
%
% \DescribeMacro{\version\marg{version}} Define the document version
% to be the string \meta{version}.  This string could include a number
% for major/minor revision as well as an alphanumeric tag for variant
% label, or course section.
%
% This version label is typeset along the right margin of the page
% along with the date and time the document was typeset.
%
% \DescribeMacro{\changes\marg{version}{date}{change}} records changes
% as in a docstrip file.  As of this version this command produces no
% output.
%
% \section{Miscellaneous eye candy}
%
% \DescribeEnv{centeredpage}
% Put the enclosed text by itself on a page, centered vertically.
% Useful for epigrams such as:
% \begin{quote}
% \textsc{Ave C\ae sar  \\ Moritvri te salutant} \\
% \emph{---Gladiator salute}
% \end{quote}
%
% \DescribeMacro{\problemfont}
% Problem statements are set in this font. When typesetting I do
%\begin{verbatim}
%\renewcommand{\problemfont}{\itshape}
%\end{verbatim}
%
% \DescribeMacro{\pagestyle\{namehead\}}
% This page style gives blanks to fill in nice info:
%
% \begin{minipage}{\textwidth}
%    \normalfont
%   Name: \hrulefill\ 
%    Section: \rule{0.5in}{0.4pt}
%    \\ \\
%    \noindent 
%    Student ID: \rule{1.5in}{0.4pt}\
%    Email: \hrulefill
%    \end{minipage}
%
% \section{Using docstrip as a preprocessor}
%
% Here is how I currently do separate modes of an exam in the same file.  The master exam file has a |.dtx|
% extension.  Versions are enclosed by \texttt{\%<*$\left<\textsf{mode}\right>$>\dots\%</$\left<\textsf{mode}\right>$>} % docstrip blocks.  
% At the top of the source file is the driver block:
%	\begin{verbatim}
%	%<*driver>
%	\input docstrip.tex
%	\askforoverwritefalse
%	\generate{\file{\jobname.tex}{\from{\jobname.dtx}{exam}}
%	         \file{\jobname-draft.tex}{\from{\jobname.dtx}{exam,draft}}
%	          \file{\jobname-pub.tex}{\from{\jobname.dtx}{exam,public}}
%	          \file{\jobname-sol.tex}{\from{\jobname.dtx}{exam,solutions,public}}
%	          \file{\jobname-key.tex}{\from{\jobname.dtx}{exam,solutions,rubric}}          
%	          \file{\jobname-all.tex}{\from{\jobname.dtx}{exam,solutions,rubric,assessment}}          
%	          \file{\jobname.ins}{\from{\jobname.dtx}{driver}}}
%	\endbatchfile
%	%</driver>
%	\documentclass[titlepage]{lexam}
%	...
%	\end{verbatim}
% 
% Not all |\file| commands may be active.
%
% Then I define my commands, using the docstrip mode specifications to undo some if I need to.  
%
% One pass of \LaTeX\ on the |.dtx| file produces all the versions.  Then I \LaTeX\ one of 
% them to get the file
%
% Advantages:
%	\begin{itemize}
%	\item Completely flexible
%	\item Each PDF file has its own name
%	\end{itemize}

%	Disadvantages:
%	\begin{itemize}
%	\item Two passes of \LaTeX\ needed.  In TeXShop, I can hit command-T on the source file, then switch to the PDF of the target version and hit command-T again.  The \LaTeX window for the target version is minimize (and left out-of-date).
%	\item Debugging is a little complicated.  Because of the above method, you need to reopen the target \LaTeX\ file to find the line number of the error
%	\end{itemize}
%
%	Possible fixes:
%	\begin{itemize}
%	\item Use one of the packages available for conditional compilation.  See
%	\url{http://www.cs.sunysb.edu/documentation/latex/help/faq/uktug-faq/FAQ191.html}
%	\end{itemize}
% 
% I think the ideal solution might still use DocStrip on some level, but have no doc blocks
% after the beginning.
%
%
%
% \StopEventually{\PrintChanges\PrintIndex}
%
% \section{Implementation}
%
% \subsection{Document Classes}
% \subsubsection{Prerequisites}
%
% These is all derived from the \LaTeXClass{article} class.
%    \begin{macrocode}
%<*examclass|quizclass|worksheetclass>
\LoadClass{article}
%    \end{macrocode}
% We use the package \LaTeXPackage{amsthm} because it gives us
% the ability to define nice environments like ``hint'' and ``solution''
%    \begin{macrocode}
\RequirePackage{amsthm}
%    \end{macrocode}
% This package enables the \LaTeXEnvironment{comment} environment\dots very useful.
%    \begin{macrocode}
\RequirePackage{verbatim}
%    \end{macrocode}
%
% I like to print solutions in a different color.  This requires a lot of |\normalcolor| commands to switch it off, which have to go at the class level.
%    \begin{macrocode}
\RequirePackage{xcolor}
%    \end{macrocode}
%
% There are several pieces to an exam: the problems, the solutions, the grading key, and 
% notes in case the exam is repeated.   The \LaTeXPackage{versions} package manages the 
% conditional compiling of code.  It is not in the te\TeX\ distribution but it is in \TeX{}live and on CTAN.
%    \begin{macrocode}
\RequirePackage{versions}
%    \end{macrocode}%
%
% For programming logic.
%    \begin{macrocode}
\RequirePackage{ifthen}
%    \end{macrocode}
%
% Nice helpers for fancy headers
%    \begin{macrocode}
\RequirePackage{fancyhdr}
%    \end{macrocode}
%


% \subsubsection{Options}
%
% \DescribeMacro{bigmath}
% \DescribeMacro{nobigmath}
% Makes all math big to be easily read (paper is highly overvalued
% when typesetting an exam).  Default is active.
%    \begin{macrocode}
\DeclareOption{bigmath}{\everymath={\displaystyle}}
\DeclareOption{nobigmath}{\everymath={}}
%    \end{macrocode}
% \DescribeMacro{draft}
% \DescribeMacro{nodraft}
% Puts ``DRAFT'' across each page and activates some commands for putting notes
% and marginalia on a draft.  Default is inactive.
%    \begin{macrocode}
\newboolean{draftmode}
\DeclareOption{draft}{\setboolean{draftmode}{true}}
\DeclareOption{nodraft}{\setboolean{draftmode}{false}}
%    \end{macrocode}
%
% \DescribeMacro{solutions}
% \DescribeMacro{nosolutions}
% Enables/Disables the typesetting of solutions.
% Default is inactive
%    \begin{macrocode}
\newboolean{solutionsmode}
\DeclareOption{solutions}{\setboolean{solutionsmode}{true}}
\DeclareOption{nosolutions}{\setboolean{solutionsmode}{false}}
%    \end{macrocode}
%
% \DescribeMacro{rubrics}
% \DescribeMacro{norubrics}
% Enables/Disables the typesetting of rubrics.
% Default is inactive
%    \begin{macrocode}
\newboolean{rubricsmode}
\DeclareOption{rubrics}{\setboolean{rubricsmode}{true}}
\DeclareOption{norubrics}{\setboolean{rubricsmode}{false}}
%    \end{macrocode}
%
% \DescribeMacro{notes}
% \DescribeMacro{nonotes}
% Enables/Disables the typesetting of notes.
% Default is inactive
%    \begin{macrocode}
\newboolean{notesmode}
\DeclareOption{notes}{\setboolean{notesmode}{true}}
\DeclareOption{nonotes}{\setboolean{notesmode}{false}}
%    \end{macrocode}
%
% Other options get passed to the parent class.
%    \begin{macrocode}
\DeclareOption*{\PassOptionsToClass{\CurrentOption}{article}}
%    \end{macrocode}
%
% The bigmath and nodraft options are default.
%    \begin{macrocode}
\ExecuteOptions{bigmath,nodraft}
\ProcessOptions
%    \end{macrocode}
%
% Gives lots of options and commands for paper size and margins.  It
% also parses the document options for ones that resemble paper size,
% such as letterpaper and a4paper.  For some reason, putting this line
% before the |\ExecuteOptions| line results in an error, so sorry this
% is out of place.
%    \begin{macrocode}
\RequirePackage{geometry}
%    \end{macrocode}

% \subsubsection{pgfkeys interface}
%
% pgfkeys rocks.  We should try to do any configuration we can through its interface.
%
%    \begin{macrocode}
\RequirePackage{pgfkeys}
\def\ngeset{\pgfqkeys{/nge}}
\def\ngestring{\pgfqkeys{/nge/strings}}
\def\ngeget#1{\pgfkeysvalueof{/nge/#1}}
%    \end{macrocode}
%
% \subsubsection{Loading other \nged packages}
%
%    \begin{macrocode}
\RequirePackage{ngeprob}
\RequirePackage{ngeform}
\RequirePackage{ngemisc}
%    \end{macrocode}
%
% \subsection{course name}
%
% \begin{macro}{\course}
%    \begin{macrocode}
\newcommand{\@shortcoursename}{\relax}
\newcommand{\course}{\@dblarg\@@course}
\def\@@course[#1]#2{
\renewcommand{\@shortcoursename}{#1}
\newcommand{\@course}{#2}}
%    \end{macrocode}
% \end{macro}
% 
%
% \subsubsection{Paragraph Types}
%
% Mathematicians like to call out paragraphs with a word like
% ``remark:'' or ``note.''  Various such are defined here.
% These use the |\newtheorem| command from the \LaTeXPackage{amsthm} package.
%
% \begin{environment}{hint}
% \changes{v0.2.2}{2002/04/11}{added the hint environment}
% For giving hints, duh. :-)
%    \begin{macrocode}
\theoremstyle{remark}
\pgfqkeys{/nge/strings}{hint/.initial=Hint,note/.initial=Note}
\newtheorem*{hint}{\pgfkeysvalueof{/nge/strings/hint}}
%    \end{macrocode}
% \end{environment}
%
% \begin{environment}{instructions}
% \changes{v0.2.2}{2002/04/11}{added}
% For giving exam instructions.  
%    \begin{macrocode}
\newenvironment{instructions}{\instructionsfont}{\normalfont}
%    \end{macrocode}
% \end{environment}
% \begin{macro}{\instructionsfont}
% \changes{v0.2.2}{2002/04/11}{added}
%    \begin{macrocode}
\newcommand{\instructionsfont}{\itshape}
%    \end{macrocode}
% \end{macro}
%
% \begin{environment}{note}
%   \begin{macrocode}
\newtheorem*{note}{\pgfkeysvalueof{/nge/strings/note}}
\theoremstyle{plain}
%    \end{macrocode}
% \end{environment}
% 
%
%
% \subsubsection{Strings}
%
% These are all pretty self-explanatory.
%
% \changes{v0.5}{2006/08/14}{Added additional sentence to exam@pageisblankname}
% \changes{v0.5}{2006/08/14}{Added exam@continuedname}
% \changes{v0.5}{2006/08/15}{Renamed 
% \texttt{\bslash exam@$\left<\right.$string$\left.\right>$name}
% $\rightarrow$ \texttt{\bslash nge@string@$\left<\right.$string$\left.\right>$}}
% \changes{v3.0-dev}{2009/04/02}{Using the pgf key registry}
%    \begin{macrocode}
\pgfqkeys{/nge/strings}{%
    pageisblank/.initial={This page intentionally left blank. \\
    You can use it for scratch work.  Please do not remove it.},
    name/.initial=Name,
    section/.initial=Section,
    SID/.initial=ID\#,
    email/.initial=email,
    signature/.initial=Signature,
    draft/.initial=DRAFT,
    note/.initial=Note,
    hint/.initial=Hint
    }
%    \end{macrocode}
%
% \subsubsection{Pagestyles}
%
% \begin{TODO}
% This probably only needs to go in the exam class.
% \end{TODO}
%
% \begin{macro}{\ps@namehead}
% This is for the title page.  It gives some blanks for filling-in
% student information and signature.
%    \begin{macrocode}
\newcommand{\ps@namehead}{%
  \renewcommand{\@oddhead}{%
    \begin{minipage}{\textwidth}
    \normalfont
    \pgfkeysvalueof{/nge/strings/name}: \hrulefill\ 
    \pgfkeysvalueof{/nge/strings/section}: \rule{0.5in}{0.4pt}
    \\ \\
    \noindent 
    \pgfkeysvalueof{/nge/strings/SID}: \rule{1.5in}{0.4pt}\
    \pgfkeysvalueof{/nge/strings/email}: \hrulefill
    \end{minipage}}
  \renewcommand{\@evenhead}{\@oddhead}
  \renewcommand{\@oddfoot}{
  \pgfkeysvalueof{/nge/strings/signature}: \hrulefill}
\renewcommand{\@evenfoot}{}}
%    \end{macrocode}
% \end{macro}
%
% \begin{macro}{\ps@problem}
% A page style for the majority of problems.  
% This one puts the problem number in bold on the top left and top right
% of every page.  In between goes the course (short) name, the test title, and the date.
%
% \begin{TODO}
% Other page styles.  The AP test has the problem number as a leader all the way across 
% the top row.  I tried 
% \begin{verbatim}
%  \renewcommand{\@oddhead}{\Huge\bfseries\hss\leaders\hbox to 0.08\textwidth{\hss\theproblem\hss}\hfill\hss}
% \end{verbatim}
% but that didn't even out the spacing quite right.  It's also kind of ugly when it \emph{is} 
% right! 
% \end{TODO}
%
% The bottom of the page has the page number centered, surrounded by en-dashes
%    \begin{macrocode}
\newcommand{\ps@problem}{%
  \setlength{\headheight}{0.5in}
  \renewcommand{\@oddhead}{%
  \begin{minipage}{\textwidth}
    {\Huge\bfseries\theproblem}
    \hfill{\normalfont\normalsize\itshape\@shortcoursename 
    \hfill \@title \hfill \@date}
    \hfill{\Huge\bfseries\theproblem}
  \\[-0.2ex]\rule{\textwidth}{0.4pt}
  \end{minipage}}
  \renewcommand{\@evenhead}{\@oddhead}
  \renewcommand{\@oddfoot}{{%
  \normalfont\normalsize\hfill--\thepage--\hfill}}
  \renewcommand{\@evenfoot}{\@oddfoot}}
%    \end{macrocode}
% \end{macro}
% 
% But at any rate, we will start with an empty page style.
%    \begin{macrocode}
\AtBeginDocument{\pagestyle{empty}}
%    \end{macrocode}
%
% \subsubsection{Document Modes}
%
%
% \subsubsection{Draft Mode}
%    \begin{macrocode}
\ifthenelse{\boolean{draftmode}}
{\message{(* draft mode on *)}
\RequirePackage{eso-pic}
\RequirePackage{graphicx}
\definecolor{draftblue}{rgb}{0.74,0.75,0.99}
\RequirePackage{type1cm}
\makeatletter
\def\stripAtPt{1}
\AddToShipoutPicture{%
     \setlength{\@tempdimb}{.5\paperwidth}%
     \setlength{\@tempdimc}{.5\paperheight}%
     \setlength{\unitlength}{\@tempdimb}%
     \put(1,1.5){%
 \makebox(0,0){\rotatebox{45}{{\textcolor[gray]{0.75}{\tt\fontsize{5cm}{5cm}\selectfont{\pgfkeysvalueof{/nge/strings/draft}}}}}}}}
\newcommand{\draftnote}[1]{{{\color{draftblue}\begin{note} #1\end{note}}}}
\newcommand{\draftmarginalnote}[1]{\marginpar{\color{draftblue}#1}}}
%    \end{macrocode}
% In non-draft mode these commands are disabled by making them ``eat'' their arguments.
%    \begin{macrocode}
{\let\draftnote=\gobble
\let\draftmarginalnote=\gobble}
%    \end{macrocode}
%
% \subsubsection{Solutions Mode}
% In the solutions version
% \begin{itemize}
% \item The contents of the \LaTeXEnvironment{inexamonly} environment are ignored
% \item The contents of the \LaTeXEnvironment{insolutions} environment are passed along without change
% \item The contents of the \LaTeXEnvironment{solution} environment
% are typeset solutions in red and with a QED symbol at the end.
% \end{itemize}
% First we have to initialize the environments
%    \begin{macrocode}
\RequirePackage{amssymb}
\newenvironment{solution}{\relax}{\relax}
\ifthenelse{\boolean{solutionsmode}}
{\message{(* solutions mode on *)}
\definecolor{darkred}{rgb}{0.75,0,0}
\newcommand{\solutionfont}{\color{darkred}\normalfont}
\AtBeginDocument{\renewcommand{\problemfont}{\itshape}}
\examformshowanswers
%    \end{macrocode}
% \begin{environment}{solution}
% The solution environment. Todo:
%	\begin{itemize}
%	\item When pagebreak happens before or during a solution, the problem text is in the 
% solution color!
%	\item Pagebreaks before solutions should be discouraged (better to break before the problem)
%	\end{itemize}
%    \begin{macrocode}
\renewenvironment{solution}[1][Solution]{%
	\renewcommand{\qedsymbol}{{\Large$\blacktriangle$}}
	\nopagebreak[3]
	\begingroup\solutionfont 
	\begin{proof}[#1]}{\end{proof}\endgroup}
%    \end{macrocode}
% \end{environment}
%    \begin{macrocode}
\gdef\problemfont{\itshape}
\global\let\examvfill=\relax
\global\let\examclearpage=\relax
%    \end{macrocode}
% \begin{environment}{insolutions}
%    \begin{macrocode}
\includeversion{insolutions}
%    \end{macrocode}
% \end{environment}
%
% \begin{environment}{inexamonly}
%    \begin{macrocode}
\excludeversion{inexamonly}
}
%    \end{macrocode}
% \end{environment}
%
% In the \textbf{absence} of solutions mode, we disable the 
% \LaTeXEnvironment{solutions} environment by letting its start and end commands
% be those of \LaTeXEnvironment{comment}.
%    \begin{macrocode}
{\message{(* solutions mode off *)}
\let\solution=\comment
\let\endsolution=\endcomment
\includeversion{inexamonly}
\excludeversion{insolutions}
%    \end{macrocode}
%
% These commands are there to put space into the document for students to write. 
% But they really only need to be available in the exam version.
%
% \begin{TODO}
% Perhaps this needs to be folded into the \LaTeXEnvironment{enumerate} environment inside of
% \LaTeXEnvironment{problem}?
% \end{TODO} 
%    \begin{macrocode} 
\newcommand{\examvfill}{\vfill}
\newcommand{\examclearpage}{\vfill\continued\clearpage}
}
%    \end{macrocode}
% \begin{macro}{\ifsolutions}
% A simple switch between text that should be used in solutions mode 
% and text that shouldn't.  I think that in this case both arguments are expanded,
% so it's probably not a good idea for large chunks of text.  But to switch between an empty 
% checkbox and a checked box it works well.
%    \begin{macrocode}
\newcommand{\ifsolutions}[2]{%
	\ifthenelse{\boolean{solutionsmode}}{#1}{#2}%
}
%    \end{macrocode}
% \end{macro}
% \subsubsection{Rubrics}  
% These are ``scoring guidelines'' that tell how points for each problem 
% should be awarded.
% \begin{macro}{marginalrubric}
% \begin{environment}{rubric}
%    \begin{macrocode}
\newcommand{\marginalrubric}[1]{\relax}
\newenvironment{rubric}{\relax}{\relax}
\let\rubric=\comment
\let\endrubric=\endcomment
%    \end{macrocode}
% If the rubric option is selected, activate the commands
%    \begin{macrocode}
\ifthenelse{\boolean{rubricsmode}}
{\message{(* rubrics on *)}
\global\def\rubricfont{\color{blue}}
\global\def\marginalrubric#1{\mbox{}\hspace{0pt}\marginpar{\rubricfont#1}}
\theoremstyle{remark}
\newtheorem*{rubricenv}{Rubric}
\renewenvironment{rubric}{\rubricfont\begin{rubricenv}}{\end{rubricenv}}
}{}
%    \end{macrocode}
% \end{environment}
% \end{macro}
%
% \subsubsection{Notes Assessment}
% 
% This is a place to put numerical data about how the problem tested, and how one might
% change it next time around.  We provide an assessment environment.
%    \begin{macrocode}
\newcommand{\metafootnote}[1]{\relax}
\newcommand{\marginalassessment}[1]{\relax}
\ifthenelse{\boolean{notesmode}}
{
\message{(* notes mode on *)}
\newcommand{\assessmentfont}{\color[rgb]{0.2,0.6,0}}
\renewcommand{\marginalassessment}[1]{\marginpar{\assessmentfont#1}}
\newcounter{metafootnote}
\renewcommand{\metafootnote}[1]{
\stepcounter{metafootnote}%
\def\thefootnote{\assessmentfont\roman{metafootnote}}%
{\assessmentfont\footnotemark[\value{metafootnote}]}%
\footnotetext[\value{metafootnote}]{{\assessmentfont#1\normalfont\normalcolor}%
}}
\theoremstyle{remark}
\newtheorem*{assessmentenv}{Assessment}
\newenvironment{assessment}{\assessmentfont\begin{assessmentenv}}{\end{assessmentenv}}
\includeversion{innotes}
}
{
\message{(* notes mode off *)}
\excludeversion{assessment}
\excludeversion{innotes}
}
%    \end{macrocode}
%
% \subsubsection{End of common code}
%    \begin{macrocode}
%</examclass|quizclass|worksheetclass>
%    \end{macrocode}
%
% \subsubsection{Title}
% \begin{macro}{\maketitle}
%
%   Quizzes and worksheets have a header on the top of the first page
%   for name, date, course, and assignment.  This is the format I
%   learned in third grade.  God bless you, Mrs. Benfield, wherever
%   you are.
%
%    \begin{macrocode}
%<*quizclass|worksheetclass>
\setlength{\headheight}{25pt}
\lhead{\@course \\ \@title}
\rhead{Name: 
\ifthenelse{\boolean{solutionsmode}}{%
  \underline{\raisebox{1pt}{\makebox[3in]{\solutionfont\sffamily\Large Solutions}}}
  }{% else
  \rule{3in}{0.4pt} 
}
\\ \@date}
\cfoot{\thepage}
\renewcommand{\maketitle}{\thispagestyle{fancy}}
\AtBeginDocument{\pagestyle{plain}}
%</quizclass|worksheetclass>
%    \end{macrocode}
%
%   For exams, we want a big title to go on a title page.
%
%   This is the |\maketitle| from \LaTeXClass{article} when using the
%   \LaTeXOption{titlepage} option.  The only difference is that
%   |\@author| is replaced by |\@course|.
%    \begin{macrocode}
%<*examclass>
\renewcommand{\maketitle}{
  \null\vfil
  \vskip 60\p@
  \begin{center}%
    {\LARGE \@title \par}%
    \vskip 3em%
    {\large
     \lineskip .75em%
      \begin{tabular}[t]{c}%
        \@course
      \end{tabular}\par}%
      \vskip 1.5em%
    {\large \@date \par}%       % Set date in \large size.
  \end{center}\par\vfil\null}
%</examclass>
%    \end{macrocode}
% \end{macro}
%
% \subsubsection{Document Partitioning (ngeexam  only)}
%
% Only used in multi-paged documents.
%
% \begin{macro}{\problems}
%   Signal the start of the ``main matter'' --- i.e., the problems
%    \begin{macrocode}
%<*examclass>
\newcommand{\problems}{%
  \pagestyle{problem}
  \setcounter{page}{1}}
%    \end{macrocode}
% \end{macro}
%
% \begin{macro}{\appendix}
%   Signal the end of problems and beginning of ``back matter''
%    \begin{macrocode}
\renewcommand{\appendix}{%
  \clearpage
  \pagestyle{plain}}
%    \end{macrocode}
% \end{macro}
%
% \begin{macro}{\blankpage}
%   Generate a blank page.  This is useful if you want a scratch page
%   at the end or after a long problem.
%    \begin{macrocode}
\newcommand{\blankpage}{%  
  \clearpage
  \begin{center}(\pgfkeysvalueof{/nge/strings/pageisblank})\end{center}
  \clearpage}
%</examclass>
%    \end{macrocode}
% \end{macro}
%
% \subsection{ngeprob.sty}
%
% \begin{TODO}
% marshal process package options to |\ngeset{every problem/.style}|
% \end{TODO}
%
%    \begin{macrocode}
%<*problems>

\RequirePackage{ifthen}
\RequirePackage{pgfkeys}

\pgfqkeys{/nge/strings}{points/.initial=Points}
\newif\ifnge@prob@grade@blank@left
\pgfqkeys{/nge/problem/grade blank margin}{.is choice,
  left/.code=\nge@prob@grade@blank@lefttrue,
  right/.code=\nge@prob@grade@blank@leftfalse,
  .initial=right}
\newif\ifnge@prob@ownpage
\def\nge@prob@headfont{\bfseries}
\ngeset{problem/.cd,
	points/.code={\setcounter{probpoints}{#1}},
    ownpage/.is if=nge@prob@ownpage,
    head font/.store in=\nge@prob@headfont,
    body font/.store in=\problemfont,
    head/.initial=\theproblem,
    label/.code={\label{#1}}
}
%    \end{macrocode}
%
% \subsubsection{Counters}
%
% \begin{macro}{problem}
% The problem number
%    \begin{macrocode}
\newcounter{problem}
%\setcounter{problem}{0}
%    \end{macrocode}
% \end{macro}
% \begin{macro}{probpoints}
% The number of points in a problem 
% (this counter is just a holder for an integer)
%    \begin{macrocode}
\newcounter{probpoints}
%    \end{macrocode}
% \end{macro}
% \begin{macro}{totalpoints}
% The total number of points in the exam
%    \begin{macrocode}
\newcounter{totalpoints}
%    \end{macrocode}
% \end{macro}

% \subsubsection{Table of Problems}
% This is done like a table of contents, with the |\@starttoc| command.
% \begin{macro}{\tableofproblems}
%    \begin{macrocode}
\newcommand{\tableofproblems}{\@starttoc{top}}
%    \end{macrocode}
% \end{macro}
%
% \begin{environment}{exam@tableofproblems}
% This internal environment generates the top and bottom of the table of
% problems.
%    \begin{macrocode}
\newenvironment{exam@tableofproblems}{%
  \setcounter{totalpoints}{0}
  \begin{center}
  \begin{tabular}{|c|c|c|}\hline
  Problem & Possible & Points \\
  Number  & Points   & Earned \\\hline}
{%
  \hline
  \t@problemstrut
  Total	  & \arabic{totalpoints} & \\\hline
  \end{tabular}
  \end{center}}
%    \end{macrocode}
% \end{environment}
%
% \begin{macro}{\t@problem}
% This internal macro generates a line of the table of problems
%    \begin{macrocode}
\newcommand{\t@problem}[2]{%
  \addtocounter{totalpoints}{#2}
  \t@problemstrut
  #1 & #2 & \\\hline
}
%    \end{macrocode}
% \end{macro}
% 
% \begin{macro}{\t@problemstrut}
% To make the table big enough to write in with red pen, 
% this internal macro generates some vertical space
% (was: |{$\displaystyle\vphantom{\int}$}|)
%    \begin{macrocode}
\pgfqkeys{/nge/table of problems}{row height/.initial=5ex,row depth/.initial=2ex}
\newcommand{\t@problemstrut}{%
  \rule[-\pgfkeysvalueof{/nge/table of problems/row depth}]{0pt}%
        {\pgfkeysvalueof{/nge/table of problems/row height}}}
%    \end{macrocode}
% \begin{TODO}
%   It might be nicer to specify the \emph{total} height of the row,
%   with the problem number centered in the row.
% \end{TODO}
% \end{macro}
%
% Now we add hooks to |\begin{document}| and |\end{document}|
% to open the table of problems file, write the top, write the bottom, 
% and close it.
%    \begin{macrocode}
\AtBeginDocument{%
  \addtocontents{top}{\protect\begin{exam@tableofproblems}}}
%    \end{macrocode}
% For the end-hook, we'd like to do
%\begin{center}
%|\addtocontents{top}{\protect\end{exam@tableofproblems}}|
%\end{center}
% but for some reason that doesn't do anything.  So we go one step
% deeper in the abstraction and copy the definition of 
% |\addtocontents|
%    \begin{macrocode}
\AtEndDocument{%
  \immediate\write\@auxout{\string\@writefile{top}%
  {\string\end{exam@tableofproblems}}}}
%    \end{macrocode}
%
% \subsubsection{Declaring Problems}
%
% \begin{environment}{problem}
%   An optional argument of key=value pairs.  We also support the
%   prior usage of a mandatory argument to mean the number of points.
%    \begin{macrocode}
%\newenvironment{problem}[1][]
  \@oldproblem
  \@problem}
\def\@oldproblem#1{\@@problem[points=#1]}
\def\@problem{\@ifnextchar[%}
  \@@problem
  {\@@problem[]}}
\def\@@problem[#1]{
%    \end{macrocode}
% Initialize.
%    \begin{macrocode}
	\ngeset{problem/points=0,every problem/.try}
%    \end{macrocode}
% Process options
%    \begin{macrocode}
	\ngeset{problem/.cd,#1}
%    \end{macrocode}
% Clear the page (optionally)
%    \begin{macrocode}
	\ifnge@prob@ownpage\clearpage\fi
%    \end{macrocode}
% Adjust counters according to the arguments.
%    \begin{macrocode}
	\refstepcounter{problem}
%    \end{macrocode}
% Write \LaTeX{} code to generate the table of problems to the |.top| file.
%    \begin{macrocode}
  \addtocontents{top}{\protect\t@problem{\theproblem}{\arabic{probpoints}}}
%    \end{macrocode}
% Typeset the problem number and number of points.  Switch to the
% problem font.
%    \begin{macrocode}
  \normalfont\normalcolor%
  \noindent%
  {\nge@prob@headfont\ngeget{problem/head}} %
  \normalfont (\arabic{probpoints} \ngestring{points})%
  {\nge@prob@headfont.} % save that space
  \problemfont%
}
%    \end{macrocode}
% At the end of the problem, put a grade blank in the bottom-right
% corner and switch back to the normal font. 
%    \begin{macrocode}
\def\endproblem{%
  \ifnge@prob@ownpage
  \vfill
  % format the grade blank in  bottom left/right corner of the page
    \ifnge@prob@grade@blank@left 
      \def\tmp@corner{south west}
      \def\tmp@xshift{1cm}
      \tikzstyle{grade blank}=[%
        at=(current page.south west),
        xshift=1cm,yshift=1cm,above right]
    \else
      \def\tmp@corner{south east}
      \def\tmp@xshift{-1cm}
      \tikzstyle{grade blank}=[%
        at=(current page.south east),
        xshift=-1cm,yshift=1cm,above left]
    \fi
%   could be prettier, e.g., [%text width=7cm,fill=red!20,rounded corners,
    \begin{tikzpicture}[remember picture,overlay] 
       \node[grade blank]
      { 
        \rule{0pt}{1cm}\rule{1cm}{0.5pt} / \arabic{probpoints}
      }; 
    \end{tikzpicture}
  \else
  % put the grade blank in the left/right margin
  % \vfill
  % this could perhaps be done with tikz, too.
    \normalfont\normalcolor
    \ifnge@prob@grade@blank@left\reversemarginpar\else\normalmarginpar\fi%
    \marginpar{\exam@gradeblank{\arabic{probpoints}}}
  \fi
}
%    \end{macrocode}
% \end{environment}
%
% \begin{macro}{\exam@gradeblank}
% (Internal)
% |\exam@gradeblank|\marg{n} generates the grade line
% \rule{0pt}{1cm}\rule{1cm}{0.5pt} / $n$.
%    \begin{macrocode}
\newcommand{\exam@gradeblank}[1]{%
  \center\rule{0pt}{1cm}\rule{1cm}{0.5pt} / #1}
%    \end{macrocode}
% \end{macro}
%
% \subsubsection{Problem parts}
%
%    \begin{macrocode}
\RequirePackage{enumitem}
%    \end{macrocode}

% \begin{environment}{partset}
% List of a ``bag'' of parts
%    \begin{macrocode}
\ngeset{part set/item sep/.initial=\z@}
\newenvironment{partset}[1][]{
  \ifnge@prob@ownpage
    \ngeset{part set/item sep=\fill}
  \else
    \ngeset{part set/item sep=0pt}
  \fi
  \ngeset{every part set/.try}
  \ngeset{part set/.cd,#1}
  \let\ltx@item=\item
  \def\item{\ltx@item}
  \begin{enumerate}[label=(\roman*),itemsep=\ngeget{part set/item sep}]
  }{%
  \end{enumerate}
}
%    \end{macrocode}
% \end{environment} 
%
% \begin{environment}{partsequence}
% List of a sequence of parts
%    \begin{macrocode}
\ngeset{part sequence/item sep/.initial=\z@}
\newenvironment{partsequence}[1][]{
  \ifnge@prob@ownpage
    \ngeset{part sequence/item sep=\fill}
  \else
    \ngeset{part sequence/item sep=0pt}
  \fi
  \ngeset{every part sequence/.try}
  \ngeset{part sequence/.cd,#1}
  \let\ltx@item=\item
  \def\item{\null\ltx@item}
  \begin{enumerate}[label=(\alph*),itemsep=\ngeget{part sequence/item sep}]
  }{%
  \end{enumerate}
}
%    \end{macrocode}
% \end{environment} 
%
% \subsubsection{Styling problems}
%
% \begin{macro}{\problemfont}
%    \begin{macrocode}
\gdef\problemfont{\normalfont}
%    \end{macrocode}
% \begin{TODO}
%   Use \LaTeXPackage{pgfkeys} to handle all styling
% \end{TODO}
% \end{macro}
% 
% Done with this package!
%    \begin{macrocode}
%</problems>
%    \end{macrocode}
%
%
% \subsection{ngeform.sty}
%
%    \begin{macrocode}
%<*forms>
\RequirePackage{ifthen}
\RequirePackage{tikz}
\RequirePackage{xkeyval}
\RequirePackage{xspace}
\InputIfFileExists{svgnam.def}

\makeatletter
\newboolean{examformshowanswers}
\newcommand{\examformshowanswers}{\setboolean{examformshowanswers}{true}}
\newcommand{\examformhideanswers}{\setboolean{examformshowanswers}{false}}
\examformhideanswers

% color business
\definecolor{ScantronGreen}{RGB}{41,103,87} % Scantron is probably a trademark
%    \end{macrocode}
% 
% \subsubsection{Boxes}
%
% An answer box is a box into which the student is supposed to write an answer.
%
% defaults -- leaving them @-less in case user wants to redefine
%    \begin{macrocode}
\newcommand{\answerboxanswerfont}{\normalfont\sffamily\Large}
\newcommand{\answerboxanswercolor}{FireBrick}
\newcommand{\answerboxboxcolor}{ScantronGreen}
\newcommand{\answerboxwidth}{1cm}
\newcommand{\answerboxheight}{1cm}

% keys for answerbox
%% eol - flush the box to the right of the line
\define@boolkey{answerbox}{eol}[true]{}
%% showanswer - show the answer
\define@boolkey{answerbox}{showanswer}[true]{}
%% answer - save the answer
\define@cmdkey{answerbox}{answer}{}
%% width - minimum width of the answer box
\define@cmdkey{answerbox}{width}{}
%% height - maxmimum width of the answer box
\define@cmdkey{answerbox}{height}{}
%% prompt - text that goes before the answer box
\define@cmdkey{answerbox}{prompt}{}

% These boxes could be implemented with LaTeX boxes,
% but not the bubbles. So does it make sense to partially refactor?
\def\answerbox{\@ifnextchar[\@answerbox{\@answerbox[]}}
\def\@answerbox[#1]{%
  % initialize local variables.
  %% copy the boolean examform@showanswers to answerbox@showanswer
  %% there might be a higher-level way to do this but I can't get it to work.
  %\let\ifKV@answerbox@showanswer=\ifexamformshowanswers
  \setkeys{answerbox}{answer=,
    width=\answerboxwidth,
    height=\answerboxheight,
    prompt=\relax,
    eol=false}%
  \ifthenelse{\boolean{examformshowanswers}}{%
    \setkeys{answerbox}{showanswer=true}}{%
    \setkeys{answerbox}{showanswer=false}}
  % process options
  \setkeys{answerbox}{#1}%
  \ifthenelse{\boolean{KV@answerbox@eol}}{%
    \leavevmode\unskip\penalty9999\hbox{}\nobreak\hfill}{\relax}%
  \@ifundefined{cmdKV@answerbox@prompt}{}{\cmdKV@answerbox@prompt }
  \tikz[baseline=0.5cm]{
    \node[above,rectangle,
          draw=\answerboxboxcolor,
          minimum width=\cmdKV@answerbox@width,
          minimum height=\cmdKV@answerbox@height,
          text=\answerboxanswercolor,
          font={\answerboxanswerfont}] {
      \ifthenelse{\boolean{KV@answerbox@showanswer}}{%
        \cmdKV@answerbox@answer}{\relax}
    };
  }%
}

%    \end{macrocode}
% \subsubsection{Bubbles}
%    \begin{macrocode}

\newcommand{\answerbubblecolor}{ScantronGreen}
\newcommand{\answerbubblefillcolor}{FireBrick}
\newcommand{\answerbubblefont}{\sffamily\upshape}

\define@boolkey{answerbubble}{fill}[true]{}
\define@boolkey{answerbubble}{correct}[true]{}

\newcommand{\answerbubble}[2][]{%
  % initialize variables
  \setkeys{answerbubble}{fill=false}
  % process options
  \setkeys{answerbubble}{#1}%
  \tikzstyle{answerbubble}=[
    draw=\answerbubblecolor,
    text=\answerbubblecolor,
    font=\answerbubblefont,
    minimum size=0.5cm,
    circle,
    inner sep=0pt
  ]%
  \ifthenelse{\boolean{KV@answerbubble@fill}}{%
    \tikzstyle{answerbubble}+=[fill=\answerbubblefillcolor]}{\relax}%
  \ifthenelse{\boolean{KV@answerbubble@correct}}{%
    \ifthenelse{\boolean{examformshowanswers}}{%
      \tikzstyle{answerbubble}+=[fill=\answerbubblefillcolor]}{\relax}}{\relax}
  \tikz[baseline=-0.1cm]{
    \node[answerbubble] {#2};            
        }%
}

%    \end{macrocode}
% \subsubsection{Rows of Exam Bubbles}
%    \begin{macrocode}
%
%% eol - flush the box to the right of the line
\define@boolkey{answerbubblerow}{eol}[true]{}
%% showanswer - show the answer
\define@boolkey{answerbubblerow}{showanswer}[true]{}

\newcommand{\answerbubblerow}[2][]{%
  % initialize
  \setkeys{answerbubblerow}{eol=false}
  \ifthenelse{\boolean{examformshowanswers}}{%
    \setkeys{answerbubblerow}{showanswer=true}}{%
    \setkeys{answerbubblerow}{showanswer=false}}
  % process options
  \setkeys{answerbubblerow}{#1}
  \ifthenelse{\boolean{KV@answerbubblerow@eol}}{%
    \leavevmode\unskip\penalty9999\hbox{}\nobreak\hfill}{\relax}%
  % walk down the list of choices
  \mbox{\answerbubblerow@walk#2,\relax,}
}

\def\answerbubblerow@walk#1,{%
  \ifx\relax#1\empty\else
  \answerbubblerow@do#1**\relax
  \expandafter\answerbubblerow@walk\fi
}

\def\answerbubblerow@do#1*#2*\relax{%
  % #2 is either empty or star depending on the input
  % If it's star and we're showing the answer, then fill
  \ifx*#2\relax
    \ifexamformshowanswers%
      \answerbubble[fill]{#1}%
    \else%
      \ifKV@answerbubblerow@showanswer%
        \answerbubble[fill]{#1}%
      \else% starred, but not in a show-answers mode
        \answerbubble{#1}%
      \fi%
    \fi%
  \else% not starred
    \answerbubble{#1}%
  \fi}

\newcommand{\tfline}[1][]{%
  \ifx#1T%
    \answerbubblerow[eol]{T*,F}%
  \else
    \ifx#1F%
      \answerbubblerow[eol]{T,F*}%
    \else
      \answerbubblerow[eol]{T,F}%
    \fi
  \fi%
}
%</forms>
%    \end{macrocode}
%
%
% \subsection{ngever.sty}
%
%    \begin{macrocode}
%<*versions>
\RequirePackage{eso-pic}
\RequirePackage{datetime}
\newcommand{\@version}{\relax}
%    \end{macrocode}
%\begin{macro}{version}
%    \begin{macrocode}
\newcommand{\version}[1]{\renewcommand{\@version}{#1}}
%    \end{macrocode}
%\end{macro}
%    \begin{macrocode}
\newcommand{\theversion}{\@version}
% ISO date: 2006-12-06
\newdateformat{isodate}{%
	\THEYEAR-\twodigit{\THEMONTH}-\twodigit{\THEDAY}%
}
\newsavebox{\versionbox}
\savebox{\versionbox}[0pt]{\color{gray} version \theversion, \isodate\today\ \currenttime}
\AtBeginDocument{
\AddToShipoutPicture{%
	\settimeformat{xxivtime}
  \setlength{\unitlength}{1in}
%  \ifthenelse{\isodd{\value{page}}}
  {\put(8.15,5.5){\rotatebox{-90}{\makebox[0pt]{\normalfont\color{gray} version \theversion, \isodate\today\ \currenttime}}}}%
%	{\put(0.35,5.5){\rotatebox{90}{\usebox{\versionbox}}}}%
}}
%    \end{macrocode}
% \begin{TODO}
%   \begin{itemize}
%   \item customization of the version stamp
%   \item option to make the stamp (default)
%   \end{itemize}
% \end{TODO}
%
% \begin{macro}{changes}
% Here I wanted something like what's in the ltxdoc package.  You can
% document changes in the preamble or at the point of the change.  The
% changelog is produced by a makeindex command.  But the
% implementation is pretty heavy. It changes some of the indexing
% stuff.  I can roll my own later.
%    \begin{macrocode}
\newcommand{\changes}[3]{\relax}
%</versions>
%    \end{macrocode}
% \end{macro}
%  
% \subsection{Miscellaneous}
% \begin{environment}{centeredpage}
%    \begin{macrocode}
%<*misc>
\newenvironment{centeredpage}{%
  \null\vfill}{%
  \vfill\clearpage}
\newcommand{\nameIDline}{\pgfkeysvalueof{/nge/strings/name}: \hrulefill\ \pgfkeysvalueof{/nge/strings/SID}: \rule{1.5in}{0.4pt}\par}
%    \end{macrocode}
% \end{environment}
%
% \begin{macro}{continued}
%    \begin{macrocode}
\pgfkeys{/nge/strings/continued/.initial=continued}
\newcommand{\continued}{%
    \normalmarginpar%
    \hfill\marginpar{(\pgfkeysvalueof{/nge/strings/continued})}}
%    \end{macrocode}
% \end{macro}
%</misc>
%
%
% \section{Example}
% I gave this test in March 2003.
% This exam actually needs to be refactored to \nged version 3
%    \begin{macrocode}
%<*example>
\documentclass[titlepage]{lexam}
\usepackage{amsmath}
\usepackage{amsfonts}
\usepackage{amssymb}
\usepackage{enumerate}
\usepackage{graphicx}

\title{Midterm I}
\author{Math 250:10 \\ Introduction to Linear Algebra}
\date{March 3, 2003}

\newcommand{\myvector}[1]{\mathbf{#1}}
\newcommand{\mymatrix}[1]{\mathbf{#1}}
\newcommand{\R}{\mathbb{R}}
\newcommand{\set}[1]{\left\{#1\right\}}
\newcommand{\row}[1]{\mathbf{r}_{#1}}
\newcommand{\FiniteSet}[1]{\mathcal{#1}}
\newcommand{\Span}{\operatorname{Span}}
\newcommand{\setS}{\FiniteSet{S}}


\begin{document}
\begin{titlepage}
\thispagestyle{namehead}

\maketitle

Show all of your work.  Full credit may not be given for an answer
alone.  You may use the backs of the pages or the extra pages for
scratch work.  Do not unstaple or remove pages.

This is a non-calculator exam. One side of one sheet of US letter-size
paper on which definitions, theorems, and formulas are handwritten by
you may be used.  Worked-out examples are \emph{not allowed} on the
formula sheet.  You will turn in your formula sheet with the exam.

By taking this exam you are agreeing to abide by these rules and Rutgers
University's Academic Integrity Policy.

\vfil
\end{titlepage}

\begin{centeredpage}
\tableofproblems
\end{centeredpage}

\problems

\begin{problem}{20}
Label each of the following statements as true (T) or false (F).  In
each, give a justification: If the statement is true, explain why; if
not, give a counterexample or cite a theorem which asserts the contrary.

Example:
\begin{enumerate}[(a)]
\item All vectors are matrices.
\begin{solution}
\textbf{True.}  By definition, a vector is a $m \times 1$ or $1 \times
n$ matrix.
\end{solution}
\item All matrices are invertible.
\begin{solution}
\textbf{False.}  Any $n \times n$ zero matrix is not invertible.
\end{solution}
\end{enumerate}

\begin{enumerate}[\rule{1cm}{0.5pt} \upshape (a)]
\item %1.1.1(a)
Matrices must be of the same size for them to be added.


\vfill\item %Cf. 1.2.1(d)
Every vector in $\R^2$ is a linear combination of any two other vectors.


\vfill\item %1.3.1(b)
Some systems of linear equations have exactly two solutions.


\vfill\item %1.3.1(i)
Performing an elementary row operation of the augmented matrix of a
system of linear equations produces the augmented matrix of an
equivalent system of linear equations.


\vfill\item %1.4.1(c)
The rank of a matrix is equal to the number of pivot columns in the
matrix.


\clearpage
\vfill\item %1.4.1(j)
The equation $A \myvector x = \myvector b$ is consistent if and only if
$\myvector b$ is a linear combination of the columns of $A$.


\vfill\item %1.6.1(i)
If $\setS_1$ and $\setS_2$ are finite subsets of $\R^n$ having equal spans,
then $\setS_1$ and $\setS_2$ have the same number of elements.


\vfill\item %1.7.1(b)
If the only solution to $A\myvector x = \myvector
0$ is $\myvector 0$, the rows of $A$ are linearly independent.


\vfill\item %1.7.1(j)
A set of vectors in $\R^n$ is linearly dependent if and only if one of
the vectors is a multiple of another.


\vfill\item %1.4.59
if $\myvector u$ and $\myvector v$ are both solutions to $A\myvector x =
\myvector b$, then $\myvector u + \myvector v$ is a solution to
$A\myvector x = \myvector b$.


\end{enumerate}
\end{problem}

\begin{problem}{15}
Write the parametric representation of the general solution to the given
system of linear equations.
\[
\begin{array}{*{4}{rc}r}
x_1 &+& 3x_2 &+& x_3 &+&  x_4 &=& -1 \\
      -2x_1 &-& 6x_2 &-& x_3 & &      &=& 5 \\
        x_1 &+& 3x_2 &+&2x_3 &+& 3x_4 &=& 2 \\
\end{array}
\]
\textbf{In this and any problem in which Gaussian Elimination is used,
label all row operations to receive full credit.}
\end{problem}


\begin{problem}{15}
Let
\[
\setS = \set{
\begin{bmatrix}1 \\ 0 \\ 0\end{bmatrix},
\begin{bmatrix}1 \\ 1 \\ 0\end{bmatrix},
\begin{bmatrix}1 \\ 1 \\ 1\end{bmatrix},
\begin{bmatrix}0 \\ 0 \\ 1\end{bmatrix}}.
\]
\begin{enumerate}[\upshape (a)]
\item Prove $\Span \setS = \R^3$.


\clearpage
\item Find a subset of $\setS$ which also spans $\R^3$ but which is
linearly independent.  Justify your claim.


\end{enumerate}
\end{problem}

\begin{problem}{10}
\begin{enumerate}[(a)]
\item Give an example of a $2 \times 2$ matrix $A$ for which $A \neq O$
but $A^2 = O$.


\vfill\item Give an example of a $2 \times 2$ matrix $B$ for which $A \neq I_2$
but $B^2 = I_2$.


\vfill\item Give an example of a $2 \times 2$ matrix $C$ for which $A \neq
I_2, O$ but $C^2 = C$.


\end{enumerate}
\end{problem}

\begin{problem}{10}
Let $B$ be an $m \times n$ matrix.  Prove that $BB^T$ and $B^TB$ are
square and symmetric matrices.

\end{problem}

\begin{problem}{15}
Let $U$ be an $m \times n$ matrix, and $Q$ an $n \times n$ matrix.  Let
$V= UQ$.  As usual, let
$\set{\myvector u_1, \dots, \myvector u_n}$ denote the set of columns of $U$, and
$\set{\myvector v_1, \dots, \myvector v_n}$ denote the set of columns of $V$.
\begin{enumerate}[(a)]
\item  Prove
\(
\Span \set{\myvector v_1, \dots, \myvector v_n} \subseteq
\Span \set{\myvector u_1, \dots, \myvector u_n}.
\)
\clearpage


\item Prove
\(
\Span \set{\myvector v_1, \dots, \myvector v_n} =
\Span \set{\myvector u_1, \dots, \myvector u_n}
\)
if $Q$ is invertible.

\end{enumerate}
\end{problem}

\begin{problem}{15}
Let
\[
A = \begin{bmatrix}
1 &-1 &0 &-1 & 2 \\
-1& 1 &1 &-2 & 1 \\
5 &-5 &-3& 4 & 1
    \end{bmatrix}.
\]
Find the reduced row echelon form $R$ of $A$ and an invertible matrix
$P$ such that $PA = R$.


\end{problem}

\appendix
\blankpage\blankpage
\end{document}
%</example>
%    \end{macrocode}
%
% \section{Examples/Tests}
%  
% \begin{TODO}
% Look at the PGF manual for examples of displayed-then-excecuted code.
% \end{TODO}
% 
% \subsection{Answerbox}
%
% \begin{macrocode}
%<*unittests>
% \begin{enumerate}
% \newcommand{\testopts}[1][]{%
% \item (Options: \texttt{#1}) What is the answer to life, the
%   universe, and everything? 
%   \answerbox[#1]}
% \testopts[eol=false,answer=42]
% \testopts[eol=true,answer=42,showanswer]
% \testopts[eol,width=2cm,showanswer]
% \testopts[]
% \end{enumerate}
%
% \subsection{Answerbubble}
%
% \begin{tabular}{lc}
% \verb+\answerbubble{A}+
% & \answerbubble{A} \\
% \verb+\answerbubble[fill]{B}+
% & \answerbubble[fill]{B} \\
% \verb+\answerbubble{C}+
% & \answerbubble{C} \\
% \end{tabular}
%
%
% \subsection{Rows of bubbles}
%
%
% \begin{enumerate}
% \item \verb+\answerbubblerow[eol]{A,B*,C,D}+
%             \answerbubblerow[eol]{A,B*,C,D}
% \item \verb+\answerbubblerow[showanswer]{A,B*,C,D}+
%             \answerbubblerow[showanswer]{A,B*,C,D}
% \item
%   \verb+\examformshowanswers\answerbubblerow{A,B*,C*,D}\examhideshowanswers+
%   \examformshowanswers\answerbubblerow{A,B*,C*,D}\examformhideanswers
% \item \verb+\answerbubblerow{?,!*,{*},{*}*,$\alpha$}
%</unittests>
%    \end{macrocode}
%
% \subsection{Exam stub}
%
% This is to get a minimal example for a ``something's wrong...maybe a missing 
% |\item|'' error message.
%
%    \begin{macrocode}
%<*examunit1>
\documentclass{ngeexam}
\title{Final Exam}
\course{Geometry 0}
\author{Euclid of Alexandra}
\date{July 4, -127}
\begin{document}

\begin{problem}[points=10]
foo
\end{problem}

\end{document}
%</examunit1>
%    \end{macrocode}
%
%    \begin{macrocode}
%<*examunit2>
\documentclass{ngeexam}
\title{Final Exam}
\course{Geometry 0}
\author{Euclid of Alexandra}
\date{July 4, -127}
\begin{document}

\begin{problem}[points=5,ownpage,grade blank margin=left]
foo1
\end{problem}

\end{document}
%</examunit2>
%<*examunit3>
\documentclass{ngeexam}
\title{Final Exam}
\course{Geometry 0}
\author{Euclid of Alexandra}
\date{July 4, -127}
\begin{document}

\begin{problem}[points=5,ownpage,grade blank margin=left]
foo
\end{problem}

\begin{problem}[points=5,ownpage,grade blank margin=right]
foo
\end{problem}


\end{document}
%</examunit3>
%    \end{macrocode}
%
% \Finale
